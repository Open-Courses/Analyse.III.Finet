\chapter{Le compactifié d'Alexandrov}

\section*{Motivation}

Le but de ce chapitre sera de donner des résultats sur la compactification
d'espace topologique.

Lorsque nous travaillons sur des espaces topologiques qui ne sont pas compacts,
nous souhaitons plonger ces espaces dans des espaces topologiques qui
sont compacts. On dit qu'on \textbf{compactifie}.

Dans notre cas, nous allons travailler avec des espaces localement compacts.

\section{Espace localement compact}

\begin{definition}
	Soit $(X, \tau)$ un espace topologique.

	On dit que $(X, \tau)$ est \textbf{(un espace topologique) localement
	compact} si tout point de $X$ contient un voisinage compact.
\end{definition}

\begin{remarque}
	De manière générale, quand on parle de propriété locale, c'est une propriété
	qui est vrai au voisinage de chaque point.
\end{remarque}

Naturellement, on peut se demander si la propriété d'être (globalement) compact comprend la
notion d'être localement compact. C'est évident vu la définition.

\begin{proposition}
	Soit $(X, \tau)$ un espace topologique.

	Si $(X, \tau)$ est un espace compact, alors $(X, \tau)$ est un espace
	localement compact.
\end{proposition}

\ifdefined\outputproof
\begin{proof}

\end{proof}
\fi

\begin{remarque}
	De manière générale, une propriété locale, comme définie précédemment, est
	globale.
\end{remarque}

\begin{definition}
	Soit $(X, \tau)$ un espace topologique.

	On dit que $(X, \tau)$ est \textbf{régulier} si tout point de $X$ admet un
	système fondamental de voisinages de fermés, c'est-à-dire si pour tout point
	$x \in X$, tout voisinage $V_{x}$ de $x$ contient un voisinage fermé de $x$.
\end{definition}

\begin{proposition}
	Soit $(X, \tau)$ un espace topologique.

	Si $(X, \tau)$ est compact, alors $(X, \tau)$ est régulier.
\end{proposition}

\ifdefined\outputproof
\begin{proof}

\end{proof}
\fi

\begin{theorem}
	Soit $(X, \tau)$ un espace topologique séparé. Les assertions suivantes sont
	équivalents.

	\begin{enumerate}
		\item $X$ est localement compact.
		\item Pour tout point $x \in X$, tout voisinage $V_{x}$ de $x$ est
			compris dans un voisinage \textit{compact} de $x$.
		\item Pour tout $x \in X$, $x$ admet un système fondamental de
			voisinages compacts.
		\item Pour tout $x \in X$, $x$ admet un système fondamental de
			voisinages relativement compacts.
		\item Tout point de $X$ admet un voisinage relativement compact.
		\item Tout point de $X$ admet un voisinage compact et fermé.
	\end{enumerate}
\end{theorem}

\ifdefined\outputproof
\begin{proof}

\end{proof}
\fi

\begin{proposition}
	Soit $(X, \tau)$ un espace topologique séparé localement compact.

	Alors les ouverts et fermés de $(X, \tau)$ sont localement compacts.
\end{proposition}

\ifdefined\outputproof
\begin{proof}

\end{proof}
\fi

\begin{proposition}
	Soit $(X, \tau)$ un espace topologique séparé.

	Alors tout sous-ensemble localement compact est l'intersection d'un ouvert
	et d'un fermé.
\end{proposition}

\ifdefined\outputproof
\begin{proof}

\end{proof}
\fi
