\section{Théorème de Banach-Steinhaus}

Le théorème suivant utilise la propriété de Baire, et la proposition que tout
espace de Banach est de Baire. Nous utiliserons le corollaire.

\begin{theorem} [Théorème de Banach-Steinhaus]
	Soient E et F deux espaces de Banach.
	Soit \GSsequence{T}{i}{I} $\subseteq$ \GScontinueHomo{E}{F}
	\label{banach-steinhaus}

	On a alors que si $\forall x \in E$ $sup_{i \in I}||T_{i}(x)|| < \infty$,
	alors $sup_{i \in I} ||T_{i}|| < \infty$
\end{theorem}

Avant de montrer ce théorème, expliquons ce qu'il signifie.

La prémisse demande que pour chaque élément $x$ de E, la famille $(T_{i}(x))_{i
\in I}$ est bornée.
On a alors que chaque application linéaire de la famille est bornée sur la
boule unité par une même constante. Nous savons, comme les applications sont
continues, qu'elles sont bornées sur la boule unité, et le théorème nous dit
qu'elles sont bornées par une même constante.

\begin{proof}
	
\end{proof}

Nous avons alors un corollaire se rapportant aux formes linéaires :

\begin{corollary}
	Soit \GSsequence{x^{*}}{n}{\naturel} $\subseteq$ $\GSdual{E}$, tel que pour
	tout $x \in E$, la suite $(x_{n}^{*}(x))_{n \in \naturel}$ est bornée. Alors
	la suite $(||x_{n}||)_{n \in \naturel}$ est bornée.
\end{corollary}

\begin{proof}
	Ce n'est qu'une application avec I = $\naturel$.
\end{proof}
