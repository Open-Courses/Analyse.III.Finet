\chapter{Théorème de Banach-Steinhaus}

Le théorème suivant utilise la propriété de Baire, et la proposition que tout
espace de Banach est de Baire. Nous utiliserons le corollaire.

\begin{theorem} [Théorème de Banach-Steinhaus]
	Soient E et F deux espaces de Banach.
	Soit $\GSsequence{T}{i}{I} \subseteq \GScontinueHomo{E}{F}$
	\label{banach-steinhaus}

	On a alors que si $\forall x \in E$ $sup_{i \in I}||T_{i}(x)|| < \infty$,
	alors $sup_{i \in I} ||T_{i}|| < \infty$
\end{theorem}

Avant de montrer ce théorème, expliquons ce qu'il signifie.

La prémisse demande que pour chaque élément $x$ de E, la famille $(T_{i}(x))_{i
\in I}$ est bornée.
On a alors que chaque application linéaire de la famille est bornée sur la
boule unité par une même constante. Nous savons, comme les applications sont
continues, qu'elles sont bornées sur la boule unité, et le théorème nous dit
qu'elles sont bornées par une même constante.

\ifdefined\outputproof
\begin{proof}
	Le théorème est une conséquence du théorème de Baire. En effet, $E$ et $F$
	(et donc $\mathcal{L}(E, F)$) sont des espaces de Baire car sont complets.

	Pour utiliser la propriété de Baire, nous allons construire une suite de
	fermé dont l'union recouvre tout $E$. On utilisera alors le
	corollaire~\ref{corollary_baire_int_non_empty} qui dit qu'un des fermés est
	d'intérieur non vide.

	Posons $E_{n} = \left\{ x \in E \, | \, \forall i \in I, \, ||T_{i}(x)||_{F}
\leq n \right\}$. On a bien que l'union des $E_{n}$ recouvre $E$.

On a alors $E_{n_{0}}$ d'intérieur non vide. Soit $x_{0} \in int({E_{n_{0}}})$.
Comme $int({E_{n_{0}}})$ est ouvert, elle est un voisinage de chacun de ses
points et donc on a qu'il existe $r > 0$ tel que $B(x_{0}, r)
\subseteq int({E_{n_{0}}}) \subseteq E_{n_{0}}$. Par définition de $E_{n_{0}}$,
on a $\forall w \in B(x_{0}, r), \, \forall i \in I, \, ||T_{i}(w)|| \leq n_{0}$.

	Soit $w \in B(x_{0}, r)$. On peut reécrire $w = x_{0} + r e$ où $||e|| <
	1$. Donc, $\forall i \in I, \, ||T_{i}(x_{0} + re)||_{F} = ||T_{i}(x_{0}) +
	rT_{i}(e)||_{F} \leq n_{0}$. 

	On en déduit $||T_{i}(e)||_{F} \leq \displaystyle \frac{1}{r} (n_{0} +
	||T_{i}(x_{0})||_{F})$.

	En se rappelant que $||T_{i}(x_{0})||_{F} \leq \sup_{i \in I}
	||T_{i}(x_{0})||_{F} = C$, qui est fini par hypothèse, et que $e$ parcourt toute
	la boule unité quand on parcours tous les $w \in B(x_{0}, r)$, on a
	$||T_{i}|| \leq \displaystyle \frac{1}{r} (n_{0} + C)$ pour tout $i$ dans
	$I$.

	Comme plus rien ne dépend de $I$, on a $\sup_{i \in I}||T_{i}|| < \infty$ 
\end{proof}
\fi

Nous avons alors un corollaire se rapportant aux formes linéaires:

\begin{corollary}
	Soit $\GSsequence{x^{*}}{n}{\naturel} \subseteq \GSdual{E}$, tel que pour
	tout $x \in E$, la suite ${(x_{n}^{*}(x))}_{n \in \naturel}$ est bornée. Alors
	la suite ${(||x_{n}||)}_{n \in \naturel}$ est bornée.
\end{corollary}

\ifdefined\outputproof
\begin{proof}
	Ce n'est qu'une application avec I = $\naturel$.
\end{proof}
\fi
