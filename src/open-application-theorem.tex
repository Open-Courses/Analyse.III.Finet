\chapter{Théorème de l'application ouverte}

Définissons d'abord ce qu'est une application ouverte.

Nous avons par définition qu'une application $f$ entre deux espaces topologiques
$X$ et $Y$ est continue si pour tout ouvert $O$ de $Y$, $f^{-1}(O)$ est un
ouvert de $X$.

Nous définissons alors dans l'autre sens la propriété d'application ouverte :

\begin{definition}
	Soient $X$ et $Y$ deux espaces topologiques, et $\GSfunction{f}{X}{Y}$. Alors
	f est ouverte si pour tout ouvert $O$ de $X$, $f(O)$ est un ouvert de $Y$.
\end{definition}

%\begin{proposition}
%	Toute application non continue est ouverte.
%\end{proposition}

%\ifdefined\outputproof
%\begin{proof}
%
%\end{proof}
%\fi

\begin{proposition}
	L'ensemble des applications ouvertes de $E$ dans $F$, noté $O(E, F)$, est
	ouvert dans $\GScontinueHomo{E}{F}$.
\end{proposition}

\ifdefined\outputproof
\begin{proof}

\end{proof}
\fi

Le théorème de l'application ouverte nous dit:

\begin{theorem} [Théorème des applications ouvertes]
\label{theorem_open_application}
	Soit $E$ et $F$ deux espaces de Banach, et $T \in \GScontinueHomo{E}{F}$ tel
	que T est surjective.

	Alors T est ouverte.
\end{theorem}

\ifdefined\outputproof
\begin{proof}

\end{proof}
\fi

Pour rappel, on appelle homéomorphisme une application linéaire continue bijective
tel que sa réciproque est aussi linéaire continue.

Par le théorème~\ref{theorem_open_application}, on en déduit les corollaires
suivants:

\begin{corollary}
	Toute application linéaire continue bijective entre deux espaces de Banach
	est un homéomorphisme.
\end{corollary}

\ifdefined\outputproof
\begin{proof}
	On a bien que $f^{-1}$ est linéaire, il reste alors à montrer que $f^{-1}$
	est continue, c'est dire que $f(O)$ est un ouvert de $F$ pour tout ouvert
	$O$ de $E$, c'est-à-dire que $f$ est ouverte.
	Par~\ref{theorem_open_application}, on a $f$ ouverte car $f$ est bijective.
\end{proof}
\fi

\begin{corollary}
	Soient deux normes $\GSnormeDef{.}{1}$ et $\GSnormeDef{.}{2}$ sur un espace de
	Banach $E$ tel que $\exists C > 0$ tel que $\forall x \in E
	\GSnormeDef{x}{1} \leq C \GSnormeDef{x}{2}$, et que
	$\GSnormedSpace{E}{\GSnormeDef{.}{1}}$ et
	$\GSnormedSpace{E}{\GSnormeDef{.}{2}}$
	sont des espaces de Banach.

	Alors les normes sont équivalentes.
\end{corollary}

\ifdefined\outputproof
\begin{proof}
	Prenons l'application identité $\GSfunction{Id_{E}}{(E, ||.||_{2})}{(E,
	||.||_{1})}$. L'hypothèse nous dit que $Id_{E}$ est continue. De plus,
	$Id_{E}$ est linéaire bijective. On a alors que
	$\GSfunction{Id_{E}}{(E, ||.||_{1})}{(E, ||.||_{2})}$ est linéaine
	continue qui nous donne $\exists M > 0, \, \forall x \in E, \,
	||x||_{2} \leq M \, ||x||_{1}$. On a donc bien que les normes sont
	équivalentes.
\end{proof}
\fi

\begin{corollary}
	Soit $E$ et $F$ des espaces de Banach.
	Soit $\GSsequence{T}{n}{\naturel} \subseteq \GScontinueHomo{E}{F}$. On
	construit $\GSfunction{T}{E}{F}$ : $T(x) = lim_{n \rightarrow \infty}
	T_{n}(x)$. Alors pour tout $K$ compact, $T_{n} \rightarrow T$ de manière
	uniforme.
\end{corollary}
