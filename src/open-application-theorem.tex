\chapter{Théorème de l'application ouverte}

Définissons d'abord ce qu'est une application ouverte.

Nous avons par définition qu'une application $f$ entre deux espaces topologiques
$X$ et $Y$ est continue si pour tout ouvert $O$ de $Y$, $f^{-1}(O)$ est un
ouvert de $X$.

Nous définissons alors dans l'autre sens la propriété d'application ouverte :

\begin{definition}
	Soient $X$ et $Y$ deux espaces topologiques, et \GSfunction{f}{X}{Y}. Alors
	f est ouverte si pour tout ouvert $O$ de $X$, $f(O)$ est un ouvert de $Y$.
\end{definition}

%\begin{proposition}
%	Toute application non continue est ouverte.
%\end{proposition}

\begin{proof}
	
\end{proof}

\begin{proposition}
	L'ensemble des applications ouvertes de $E$ dans $F$, noté $O(E, F)$, est
	ouvert dans \GScontinueHomo{E}{F}.
\end{proposition}

\begin{proof}
	
\end{proof}

Le théorème de l'application ouverte nous dit :

\begin{theorem} [Théorème des applications ouvertes]
	Soit $E$ et $F$ deux espaces de Banach, et $T \in$ \GScontinueHomo{E}{F} tel
	que T est surjective.

	Alors T est ouverte.
	\label{open-application-theorem}
\end{theorem}

\begin{proof}
	
\end{proof}

Pour rappel, on appelle application bicontinue une application bijective
continue tel que sa réciproque est aussi continue.

Par le théorème \ref{open-application-theorem}, on en déduit les corollaires
suivants :

\begin{corollary}
	Toute application bijective continue entre deux espaces de Banach est bicontinue.
\end{corollary}

\begin{proof}
	
\end{proof}

\begin{corollary}
	Soient deux normes \GSnormeDef{.}{1} et \GSnormeDef{.}{2} sur E tel que
	$\exists C > 0$ tel que $\forall x \in E$ \GSnormeDef{x}{1} $\leq$
	C \GSnormeDef{x}{2}, et que \GSnormedSpace{E}{\GSnormeDef{.}{1}} et
	\GSnormedSpace{E}{\GSnormeDef{.}{2}} sont des espaces de Banach.
	
	Alors les normes sont équivalentes.
\end{corollary}

\begin{proof}
	
\end{proof}
