\chapter{Topologie (pré-)faible et applications}

%\section{Rappel : Topologie initiale}

%Prenons une famille
%$( (E_{i}, \tau_{i}) )_{i \in I}$ d'espace topologique ainsi qu'une famille de fonctions $\GSsequence{f}{i}{I}$ tel que
%$\GSfunction{f_{i}}{E}{E_{i}}$ où E est un ensemble.

%Nous allons construire une topologie sur E, appelée \textbf{topologie initiale}
%et notée $\tau$, telle que les applications $f_{i}$ sont continues avec la
%topologie $\tau$. Il faut
%pour cela que pour chaque $i \in I$, et pour chaque ouvert $O_{i} \in \tau_{i}$,
%$f_{i}^{-1}(O_{i}) \in \tau$.

%Nous posons une condition supplémentaire sur $\tau$ qui est qu'elle est la
%topologie la moins fine qui rend les $f_{i}$ continues, c'est-à-dire que si on
%prend une autre topologie $\tau^{*}$ sur E qui rend les $f_{i}$ continues, chaque
%ouvert de $\tau$ doit être dans $\tau^{*}$.

%On construit $\tau$ comme l'intersection finie des $f_{i}^{-1}(O_{i})$ où $O_{i} \in
%\tau_{i}$.

%\begin{exemple}
	%La topologie produit est un cas particulier de topologie initiale où $E =
	%\displaystyle \prod_{i \in I} E_{i}$, et où les $f_{i}$ sont les projections.
%\end{exemple}

%\section{Topologie faible}

%Soient $\GSnormedSpace{E}{\GSnormeDef{.}{E}}$ un espace vectoriel normé et
%$\GSnormedSpace{\GSdual{E}}{\GSnormeDef{.}{\GSdual{E}}}$ son dual.

%Soient $x \in E$ et les fonctions

%\begin{equation}
	%\phi_{x} : E^{*} \rightarrow \mathbb{K} : x^{*} \rightarrow \abs{x^{*}(x)}
%\end{equation}

%La topologie faible est la topologie initiale avec les fonctions $\phi_{x}$, les
%espaces vectoriels $E_{i} = E^{*}$, et l'espace de départ est $E$. Cette
%topologie est notée $\GSweakTopo{E}$.

%Les ouverts sont donc de la forme:

%\begin{equation}
	%\displaystyle \bigcap_{\substack{x \in J \\ J \subseteq E \\ J finie}} \phi_{x}^{-1}
	%(O_{x})
%\end{equation}
%où $O_{x}$ est un ouvert de $\mathbb{K}$.

%Comme les boules centrées à l'origine $B(0, \epsilon)$ forment un système
%fondamental de voisinage de $0$ dans $\mathbb{K}$, on peut
%supposer, sans perte de généralité, que $O_{x}$ est une boule.

%$J$ étant fini, on se permet de noté ses éléments $x_{1}, \cdots, x_{n}$. On se
%retrouve alors avec la forme suivante.

%\begin{equation}
	%\displaystyle \bigcap_{i \in \GSset{1, \cdots, n}}
%\phi_{x_{i}}^{-1}(]-\epsilon_{i}, \epsilon_{i}[)
%\end{equation}

%c'est-à-dire

%\begin{equation}
	%\GSsetDef{x \in E}{ \forall i \in \GSset{1, \cdots, n} \abs{\phi_{x_{i}}(x)} <
	%\epsilon_{i}}
%\end{equation}

%En se rappelant la définition des $\phi_{x_{i}}$, on a

%\begin{equation}
	%\GSsetDef{x \in E}{ \forall i \in \GSset{1, \cdots, n} \abs{x_{i}^{*}(x)} <
	%\epsilon_{i}}
%\end{equation}

%et si on pose $\epsilon = \min(\epsilon_{1}, \cdots, \epsilon_{n})$, on obtient

%\begin{equation}
	%\GSsetDef{x \in E}{ \forall i \in \GSset{1, \cdots, n} \abs{x_{i}^{*}(x)} <
	%\epsilon}
%\end{equation}

%On pose alors

%\begin{equation}
	%V_{\epsilon, x_{1}^{*}, \ldots, x_{n}^{*}} = \GSsetDef{x \in E}{ \forall i
		%\in \GSset{1, \cdots, n} \abs{x_{i}^{*}(x)} < \epsilon}
%\end{equation}



%Nous avons vu dans le chapitre sur les espaces de Banach, que dans un espace
%vectoriel normé, la boule unité est compacte ssi l'espace est de dimension
%finie (~\ref{theorem_riesz_compact}). Nous cherchons à définir une topologie sur
%$E$ tel que la boule unité est compacte pour cette topologie, et la topologie
%$\GSweakTopo{E}$ convient.

\section{Rappel}

\begin{definition}
	Soit $(X, \tau)$ un espace topologique et soit $\mathcal{B} \in
	\mathcal{P}(X)$.

	On dit que \textbf{$\mathcal{B}$} est \textbf{une base d'ouverts} si tout
	élément de $\mathcal{B}$ est un ouvert de $(X, \tau)$ et si tout ouvert de
	$(X, \tau)$ s'écrit comme une union d'éléments de $\mathcal{B}$.
\end{definition}

Nous serons menés à travailler avec d'un côté la topologie engendrée par la
norme $\GSnormeDef{.}{E}$, et d'un autre avec la topologie faible
$\GSweakTopo{E}$. Nous serons menés à parler de convergence, d'ouverts/fermés et
nous aurons besoin de distinguer quand nous parlons pour la topologie faible ou
non.

Pour cela, nous allons introduire des noms pour distinguer ces différences.


\begin{definition}
	Soient $\GSnormedSpace{E}{\GSnormeDef{.}{E}}$ un espace vectoriel normé,
	$\GSdual{E}$ son dual et $\GSweakTopo{E}$ sa topologie faible.

	On dit qu'un ouvert (resp un fermé) de $(E, \GSweakTopo{E})$ est
	\textbf{faiblement ouvert} ou \textbf{$\sigma$-ouvert} (resp
	\textbf{faiblement fermé} ou \textbf{$\sigma$-fermé}).

	L'adhérence d'un ensemble $A$ dans $(E, \GSweakTopo{E})$ sera notée
	$\adh{A}^{\sigma}$ et l'intérieur $\interior{A}^{\sigma}$.
\end{definition}

\subsection{Convergence pour la topologie faible}
Quels sont les liens entre la topologie faible et la topologie sur $\GSdual{E}$
qui a servi à la construire?

D'abord, regardons la définition de convergence dans $\GSweakTopo{E}$.
\begin{definition}
	Soit une suite $\GSsequence{x}{n}{\naturel} \subseteq E$, et $x \in E$.
	On dit que $\GSsequence{x}{n}{\naturel}$ \textbf{converge faiblement} vers $x$
	pour $\GSweakTopo{E}$, et on note
	\begin{equation}
		x_{n} \xrightarrow{\GSweakTopo{E}} x
	\end{equation}
	si pour tout voisinage $V(x)$ de $x$,
	\begin{equation}
		\exists n_{0} \, \forall n \geq n_{0}, x_{n} \in V(x)
	\end{equation}
	ou de manière équivalente, pour tout voisinage $V(0)$ de $0$.
	\begin{equation}
		\exists n_{0} \, \forall n \geq n_{0}, x_{n} - x \in V(0)
	\end{equation}
\end{definition}

Il est parfois dure de travailler avec cette définition, et on a alors une
équivalence plus facile à utiliser.

\begin{proposition}
	Soient $\GSnormedSpace{E}{\GSnormeDef{.}{E}}$ un espace vectoriel normé, $\GSdual{E}$ son dual et
	$\GSweakTopo{E}$ sa topologie faible.
	Les assertions suivantes sont équivalentes
	\begin{enumerate}
		\item $x_{n} \xrightarrow{\GSweakTopo{E}} x$
		\item $\forall x^{*} \in \GSdual{E}$, $x^{*}(x_{n}) \rightarrow x^{*}(x)$.
	\end{enumerate}
\end{proposition}

\ifdefined\outputproof
\begin{proof}

\end{proof}
\fi

\subsection{Lien entre la topologie de la norme et la topologie faible}

Continuons à donner des liens liant la topologie faible et la norme sur $E$.

\begin{proposition}
	Soit $\GSnormedSpace{E}{\GSnormeDef{.}{E}}$ un espace vectoriel normé.
	Soit $C$ convexe non vide de $E$ (ça ne dépend pas de la topologie).
	Alors les assertions suivantes sont équivalentes.
	\begin{enumerate}
		\item $C$ est fermé dans $\GSnormedSpace{E}{\GSnormeDef{.}{E}}$
		\item $C$ est faiblement fermé (ie fermé dans $(E, \GSweakTopo{E})$.
	\end{enumerate}
\end{proposition}

\ifdefined\outputproof
\begin{proof}

\end{proof}
\fi

Nous pouvons nous demander à quoi ressemblent les ouverts de la topologie
$\GSweakTopo{E}$. En fait, en dimension finie, la topologie est celle de la norme.

\begin{proposition}
	Soit $\GSnormedSpace{E}{\GSnormeDef{.}{E}}$ un espace vectoriel normé de
	dimension finie, alors $\tau_{\GSnormeDef{.}{E}} = \GSweakTopo{E}$.
\end{proposition}

\ifdefined\outputproof
\begin{proof}

\end{proof}
\fi

Etudions ce qui se passe en dimension infinie.

Nous savons que pour la norme, $S_{E}$, la sphère unitée, est fermée, et donc
qu'elle vaut son adhérence.  Dans le cas $E$ fini, la sphère est également
faiblement fermée par la proposition précédente. En dimension infinie, nous
obtenons quelque chose de surprenant. En effet, $\overline{S_{E}}^{\sigma}$,
c'est-à-dire l'adhérence de la sphère unitée par rapport à la topologie faible,
vaut la boule unité fermée pour la norme.
En d'autres termes,

\begin{proposition}
	Soit $\GSnormedSpace{E}{\GSnormeDef{.}{E}}$ un espace vectoriel normé de
	dimension infinie. Alors
	\begin{equation}
		\adh{S_{E}}^{\sigma} = \GSsetDef{x \in E}{\GSnormeDef{x}{E} \leq 1}
	\end{equation}
\end{proposition}

\ifdefined\outputproof
\begin{proof}

\end{proof}
\fi

\begin{lemma}
	Soient $\GSnormedSpace{E}{\GSnormeDef{.}{E}}$ un espace vectoriel normé de
	dimension infinie et
	$\GSnormedSpace{\GSdual{E}}{\GSnormeDef{.}{\GSdual{E}}}$ son dual.

	Soient $x_{1}^{*}, \cdots, x_{n}^{*} \in \GSdual{E}$.

	Alors $x_{1}^{*}, \cdots, x_{n}^{*}$ ont une racine commune. En d'autres
	termes, il existe $y_{0}$ tel que pour tout $i$ entre $1$ et $n$,
	$x_{i}(y_{0}) = 0$.
\end{lemma}

\ifdefined\outputproof
\begin{proof}

\end{proof}
\fi

\begin{lemma}
	Soient $\GSnormedSpace{E}{\GSnormeDef{.}{E}}$ un espace vectoriel normé de
	dimension infinie, $\GSnormedSpace{\GSdual{E}}{\GSnormeDef{.}{\GSdual{E}}}$
	son dual et $\GSweakTopo{E}$ sa topologie faible.

	Soit $x$ un point de $E$.
	Tout voisinage faible de $x$ contient une droite passant par $x$.

	En particulier, tout voisinage faible n'est pas borné.
\end{lemma}

\ifdefined\outputproof
\begin{proof}

\end{proof}
\fi

Que se passe-t-il pour la boule unité ouvert, c'est-à-dire
\begin{equation}
	B_{\GSnormeDef{.}{E}}(0, 1) := \GSsetDef{x \in E}{\GSnormeDef{x}{E} < 1}
\end{equation}
?

\begin{proposition}
	Soient $\GSnormedSpace{E}{\GSnormeDef{.}{E}}$ un espace vectoriel normé,
	$\GSdual{E}$ son dual et $\GSweakTopo{E}$ sa topologie faible.

	La boule unité ouverte pour $\GSnormeDef{.}{E}$ est d'intérieur vide pour la
	topologie faible. Par conséquence, elle n'est pas $\sigma$-ouverte.

	En d'autres termes, si nous posons $U = B_{\GSnormeDef{.}{E}}(0, 1)$
	\begin{equation}
		\interior{U}^{\sigma} = \emptyset
	\end{equation}
\end{proposition}

\ifdefined\outputproof
\begin{proof}

\end{proof}
\fi

\subsection{Applications}

\begin{theorem}
	Soient $\GSnormedSpace{E}{\GSnormeDef{.}{E}}$ un espace vectoriel normé,
	$\GSdual{E}$ son dual et $\GSweakTopo{E}$ sa topologie faible.

	Alors la boule unité est faiblement compacte ie compacte dans $(E,
	\GSweakTopo{E})$
\end{theorem}

\ifdefined\outputproof
\begin{proof}

\end{proof}
\fi

\subsection{Applications dans les espaces de Banach}

\begin{proposition}
	Soit $\GSnormedSpace{E}{\GSnormeDef{.}{E}}$ un espace de Banach et soit
	$B$ un sous-ensemble non vide de $E$.

	Si pour tout $x^{*} \in E^{*}$, $x^{*}(B)$ borné dans $\real$, alors $B$ borné
	dans $E$.
\end{proposition}

\ifdefined\outputproof
\begin{proof}

\end{proof}
\fi

\begin{corollary}
	Soit $\GSnormedSpace{E}{\GSnormeDef{.}{E}}$ de Banach.
	Prenons une suite $\GSsequence{x}{n}{\naturel}$ dans $E$ faiblement
	convergente vers $x$.
	Alors 
	\begin{enumerate}
		\item $(\GSnormeDef{x_{n}}{E})_{n \in \naturel}$ est bornée
		\item $\GSnormeDef{x}{E} \leq \lim\inf \GSnormeDef{x_{n}}{E}$
	\end{enumerate}
\end{corollary}

\ifdefined\outputproof
\begin{proof}

\end{proof}
\fi

\begin{proposition}
	Soient $\GSnormedSpace{E}{\GSnormeDef{.}{E}}$ un espace de Banach,
	$\GSdual{E}$ son dual et $\GSpreweakTopo{E}$ sa topologie pré-faible.

	Soient $\GSsequence{x}{n}{\naturel}$ une suite bornée de $E$ et $x$ dans
	$E$.

	Si
	\begin{equation}
		\GSsetDef{x^{*} \in \GSdual{E}}{x^{*}(x_{n})
		\xrightarrow{\GSnormeDef{.}{\GSdual{E}}} x^{*}(x)}
	\end{equation}
	est dense dans
	$\GSdual{E}$, alors $x_{n} \xrightarrow{\GSpreweakTopo{E}} x$.
\end{proposition}

\ifdefined\outputproof
\begin{proof}

\end{proof}
\fi

\section{Topologie pré-faible}

\begin{proposition}
	Soient $\GSnormedSpace{E}{\GSnormeDef{.}{E}}$ un espace vectoriel normé,
	$\GSnormedSpace{\GSdual{E}}{\GSnormeDef{.}{\GSdual{E}}}$ son dual et soit
	$\GSpreweakTopo{E}$ sa topologie pré-faible.

	Soient $\GSsequence{x^{*}}{n}{\naturel}$ une suite de $\GSdual{E}$ et
	$x^{*}$ dans $\GSdual{E}$. Les assertions suivantes sont équivalentes.
	\begin{enumerate}
		\item $x^{*}_{n} \xrightarrow{\GSpreweakTopo{E}} x^{*}$
		\item $\forall x \in E$, $x^{*}_{n}(x) \rightarrow x^{*}(x)$.
	\end{enumerate}
\end{proposition}

\ifdefined\outputproof
\begin{proof}

\end{proof}
\fi
