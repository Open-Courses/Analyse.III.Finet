\chapter{Topologie (pré-)faible et applications}

\section{Rappel : Topologie initiale}

Prenons une famille
$\GSsequence{E}{i}{I}$ d'espace topologique, où la topologie sur $E_{i}$ est notée
$\tau_{i}$, ainsi qu'une famille de fonctions $\GSsequence{f}{i}{I}$ tel que
$\GSfunction{f_{i}}{E}{E_{i}}$ où E est un ensemble.

Nous allons construire une topologie sur E, appelée \textbf{topologie initiale}
et noté $\tau$, telle qu'elle rend les applications $f_{i}$ continues. Il faut
pour cela que pour chaque $i \in I$, et pour chaque ouvert $O_{i} \in \tau_{i}$,
$f_{i}^{-1}(O_{i}) \in \tau$.

Nous posons une condition supplémentaire sur $\tau$ qui est qu'elle est la
topologie la moins fine qui rend les $f_{i}$ continues, c'est-à-dire que si on
prend une autre topologie $\tau^{*}$ sur E qui rend les $f_{i}$ continues, chaque
ouvert de $\tau$ doit être dans $\tau^{*}$.

On construit $\tau$ comme l'intersection finie des $f^{-1}(O_{i})$ où $O_{i} \in
\tau_{i}$.

\begin{exemple}
	La topologie produit est un cas particulier de topologie initiale où $E =
	\displaystyle \prod_{i \in I} E_{i}$, et où les $f_{i}$ sont les projections.
\end{exemple}

\section{Topologie faible}

Soit $x \in E$ et $\phi_{x} : E^{*} \rightarrow \mathbb{K} : x^{*} \rightarrow
|x^{*}(x)|$.

La topologie faible est la topologie initiale avec les fonctions $\phi_{x}$, les
espaces vectoriels $E_{i} = E^{*}$, et l'espace de départ est $E$. C'est un cas
particulier de la topologie faible où $I$ couvre tout $E$. Cette topologie est
notée $\GSweakTopo{E}$.

Les ouverts sont donc:

$\left\{ x \in E \, | \, x \in \displaystyle \cap_{i = 1}^{n}
{\phi^{-1}}_{{x_{i}}}(O_{i}) \right\}$. Comme les boules centrées à l'origine
forment une base d'ouvert, on peut sans perte de généralité que $O_{i}$ est une
boule.

$\left\{ x \in E \, | \, \forall \epsilon > 0, \cap_{i = 1}^{n}|\phi_{x_{i}}(x)| \leq \epsilon| \right\}$
et en se rappelant la définition des $\phi_{x_{i}}$, on a
$\left\{ x \in E \, | \, \forall \epsilon > 0, \cap_{i = 1}^{n}|x_{i}^{*}(x)| \leq
\epsilon| \right\} = $
$\left\{ x \in E \, | \, \forall \epsilon > 0, \, \forall 1 \leq i \leq n, \, |x_{i}^{*}(x)| \leq \epsilon| \right\}$

Nous avons vu dans le chapitre sur les espaces de Banach, que dans un espace
vectoriel normé, la boule unité est compacte ssi l'espace est de dimension
finie (~\ref{theorem_riesz_compact}). Nous cherchons à définir une topologie sur
$E$ tel que la boule unité est compacte pour cette topologie, et la topologie
$\GSweakTopo{E}$ convient.

\subsection{Convergence pour la topologie faible}
Quels sont les liens entre la topologie faible et la topologie sur $\GSdual{E}$
qui a servi à la construire?

D'abord, regardons la définition de convergence dans $\GSweakTopo{E}$.
\begin{definition}
	Soit une suite $\GSsequence{x}{n}{\naturel} \subseteq E$, et $x \in E$.
	On dit que $\GSsequence{x}{n}{\naturel}$ \textbf{converge faiblement} vers $x$
	pour $\GSweakTopo{E}$, et on note $x_{n} \xrightarrow{\sigma} x$ si pour tout
	voisinage $V(x)$ de $x$ (resp $V(0)$ de $0$), $\exists n_{0}$ tel que
	$\forall n \geq n_{0}$, $x_{n} \in V(x)$ (resp $x_{n} - x \in V(0)$).
\end{definition}

Il est parfois dure de travailler avec cette définition, et on a alors une
équivalence plus facile à utiliser.

\begin{proposition}
	$x_{n} \xrightarrow{\sigma} x \Leftrightarrow \forall x^{*} \in \GSdual{E}$,
	$x^{*}(x_{n}) \rightarrow x^{*}(x)$.
\end{proposition}

\begin{proof}

\end{proof}

\subsection{Lien entre la topologie de la norme et la topologie faible}

Continuons à donner des liens liant la topologie faible et la norme sur $E$.

\begin{proposition}
	Soit $C$ convexe non vide de $E$ (ça ne dépend pas de la topologie).
	On a $C$ fermé pour la norme ssi $C$ est faiblement fermé (ie fermé pour
	$\GSweakTopo{E}$).
\end{proposition}

\begin{proof}

\end{proof}

Nous pouvons nous demander à quoi ressemblent les ouverts de la topologie
$\GSweakTopo{E}$. En fait, en dimension finie, la topologie est celle de la norme.

\begin{proposition}
	Soit $\GSnormedSpace{E}{\GSnorme{.}}$ de dimension finie, alors $\tau_{||.||}
	= \GSweakTopo{E}$.
\end{proposition}

\begin{proof}

\end{proof}

Etudions ce qui se passe en dimension infinie.

Nous savons que pour la norme, $S_{E}$, la sphère unitée, est fermée, et donc
qu'elle vaut son adhérence.  Dans le cas $E$ fini, la sphère est également
faibelemnt fermée par la proposition précédente. En dimension infinie, nous
obtenons quelque chose de surprenant. En effet, $\overline{S_{E}}^{\sigma}$,
c'est-à-dire l'adhérence de la sphère unitée par rapport à la topologie faible,
vaut la boule unité fermée pour la norme.

\begin{proposition}
	Soit $E$ de dimension infinie, et $\GSnorme{.}$ sur $E$. Alors
	$\overline{S_{E}}^{\sigma} = \left\{ x \in E \, | \, ||x|| \leq 1 \right\}$.
\end{proposition}

\begin{proof}

\end{proof}

Que se passe-t-il pour la boule $\left\{ x \in E \, | \, ||x|| < 1 \right\}$?

\begin{proposition}
	La boule unité ouverte pour $\GSnorme{.}$ est d'intérieur vide pour la
	topologie faible. Par conséquence, elle n'est pas $\sigma$-ouverte.
\end{proposition}

\begin{proof}

\end{proof}

\subsection{Applications}

\begin{theorem}
	Soit l'espace topologique $(E, \GSweakTopo{E})$, alors la boule unité
	est compacte.
\end{theorem}

\begin{proof}

\end{proof}

\subsection{Applications dans les espaces de Banach}

\begin{proposition}
	Soit $\GSnormedSpace{E}{\GSnorme{.}}$ de Banach, et un sous ensemble $B$ non
	vide de $E$.
	Si $\forall x^{*} \in E^{*}$, $x^{*}(B)$ borné dans $\real$, alors $B$ borné
	dans $E$.
\end{proposition}

\begin{proof}

\end{proof}

\begin{corollary}
	Soit $\GSnormedSpace{E}{\GSnorme{.}}$ de Banach.
	Prenons une suite $\GSsequence{x}{n}{\naturel}$ dans $E$ faiblement
	convergente vers $x$.
	Alors $\GSsequence{||x_{n}||}{n}{\naturel}$ est bornée et $\GSnorme{x} \leq
	\lim\inf \GSnorme{x_{n}}$.
\end{corollary}

\begin{proof}

\end{proof}

\begin{proposition}
	Prenons une suite bornée de $E$ ou $E$ est un espace de Banach, et $x$ dans
	$E$.
	Si $\left\{ x^{*} \in E^{*} \, | \, x^{*}(x_{n}) \rightarrow x^{*}(x)
\right\}$ est dense dans $E^{*}$, alors $x_{n} \xrightarrow{\sigma} x$.
\end{proposition}

\begin{proof}

\end{proof}

\section{Topologie pré-faible}


\begin{proposition}
	$x^{*}_{n} \xrightarrow{\sigma^{*}} x^{*} \Leftrightarrow \forall x \in \GSdual{E}$,
	$x^{*}_{n}(x) \rightarrow x^{*}(x)$.
\end{proposition}
