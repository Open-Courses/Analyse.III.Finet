\chapter{Ensemble de première catégorie d'un espace topologique}

\begin{definition}
	Soit $(X, \tau)$ un espace topologique, et $A \subset X$. On dit que $A$ est
	\textbf{nul part dense} ou \textbf{maigre} si $\int{\adh{A}} =
	\emptyset$. On la note $\mathcal{C}_{npd}$.
\end{definition}

Par exemple, avec $(\real, |.|)$, $\emptyset$, $\naturel$, $\integer$ et tous
les ensembles finis sont nul part denses.

\begin{proposition}
	La classe des ensembles nul part denses est \textbf{héréditaire},
	c'est-à-dire que $\forall A \in \mathcal{C}_{npd}$, $\forall B \subseteq A$,
	$B \in \mathcal{C}_{npd}$.
\end{proposition}

\begin{proof}
	$B \subseteq A \implies \adh{B} \subseteq \adh{A} \implies
	\int{\adh{B}} \subseteq \int{\adh{A}}$ et $\int{\adh{A}}= \emptyset$ par
	hypothèse sur $A$. Donc $\int{\adh{B}} = \emptyset$, et c'est la définition
	d'être nul part dense.
\end{proof}

\begin{proposition}
	$\mathcal{C}_{npd}$ est \textbf{stable par fermeture}, c'est-à-dire que
	$\forall A \in \mathcal{C}_{npd}$, $\adh{A} \in \mathcal{C}_{npd}$.
\end{proposition}

\begin{proof}
	$\adh{\adh{A}} = \adh{A}$, et donc, comme $A \in \mathcal{C}_{npd}$,
	$\int{\adh{\adh{A}}} = \int{\adh{A}} = \emptyset$
\end{proof}

\begin{proposition}
	$\mathcal{C}_{npd}$ est stable par union finie.
\end{proposition}

\begin{proof}
	Il suffit de montrer pour $n = 2$, le cas $n > 2$ se montre par récurrence.
	Soit $A$ et $B$ nul part dense. Il faut montrer que $A \union B$ est nul
	part dense, c'est-à-dire que $\int{\adh{A \union B}} = \emptyset$.
\end{proof}

\begin{definition}
	Soit $(X, \tau)$ un espace topologique.
	Un ensemble est de \textbf{première catégorie} s'il est une union
	dénombrable d'ensembles nul part denses. Nous notons $\mathcal{C}_{1}$ la
	classe des ensembles de première catégorie.
\end{definition}

Nous avons par exemple $\rational$ qui est de première catégorie dans $(\real,
|.|)$.

\begin{proposition}
	$\mathcal{C}_{1}$ est héréditaire et stable par union
	dénombrable.
\end{proposition}

\begin{proof}
	$\mathcal{C}_{1}$ stable par union dénombrable:

	Soit $A_{i} = \union_{n \in \naturel} A_{i, n}$ une famille dénombrable tel
	que chaque $A_{i, n}$ est nul part dense. On a bien que $\union_{i \in \naturel,
	n \in \naturel} A_{i, n}$ est une union dénombrable (union sur
	$\naturel^{2}$ qui est dénombrable), et chaque $A_{i, n}$
	est nul part dense par hypothèse.

	$\mathcal{C}_{1}$ héréditaire:
	
	Soit $A = \union_{n \in \naturel} A_{n}$ avec chaque $A_{n}$ nul part dense,
	et $B$ tel que $A \subseteq B$. On pose $B_{0} = A_{0} \union B\backslash
	A$, et $B_{n} = A_{n}$ pour $n > 0$. Comme $A_{0}$ est nul part dense que et
	$\mathcal{C}_{npd}$ héréditaire, $B_{0}$, qui comprend $A_{0}$, tel nul part
	dense. Donc $B = \union_{n \in \naturel} B_{n}$ est de première catégorie.
\end{proof}

\begin{remarque}
	$\mathcal{C}_{1}$ n'est pas stable par fermeture.
	% Donner un exemple
\end{remarque}

\begin{definition}
	Soit $(X, \tau)$ un espace topologique, et $A \subseteq X$.
	Alors $A$ est un $G_{\delta}$ (resp. $F_{\sigma}$) s'il est intersection
	dénombrable d'ouverts (resp. union dénombrable de fermés).
\end{definition}

\section{Applications dans les espaces de Baire (\ref{chapter_baire_theorem})}

Remarquons d'abord que la propriété de Baire fermé
(\ref{property_baire_closed}) peut être écrit en toute union dénombrable
de fermés nul part denses est d'intérieur vide.

\begin{proposition}
	\label{proposition_first_set_category_equiv_baire}
	Soit $(X, \tau)$ est un espace topologique.
	$(X, \tau)$ est de Baire ssi le complémentaire de tout ensemble de première
	catégorie est dense dans $X$, c'est-à-dire $\forall A \in \mathcal{C}_{1}$,
	$\adh{A^{c}} = X$.
\end{proposition}

\begin{proof}
	$(\Rightarrow)$
	Soit $S$ un ensemble de première catégorie, ie $S = \union_{n \in \naturel}
	S_{n}$ où $S_{n}$ est nul part dense. Il faut montrer que $\comp{S} =
	\inter_{n \in \naturel} \comp{(S_{n})}$ est dense dans $X$.

	(notation lourde)

	$(\Leftarrow)$
	Soit \GSsequence{F}{n}{\naturel} une suite dénombrable de fermé tel que
	$\int{F_{n}} = \emptyset$.
	Il faut montrer que $F = \union_{n \in \naturel} F_{n} = \emptyset$
	(\ref{property_baire_closed}).
	On a $F$ de première catégorie car chaque $F_{n}$ est nul part dense, et
	donc par hypothèse, $\adh{\comp{F}} = X$, c'est-à-dire $\int{F} =
	\emptyset$ (en passant au complémentaire).
\end{proof}

\begin{corollary}
	Si $(X, \tau)$ est un ensemble de Baire, alors $X \notin \mathcal{C}_{1}$.
\end{corollary}

\begin{proof}
	Si $X$ est de première catégorie, alors $\emptyset = \comp{X}$ serait dense
	dans $X$. Or $\adh{\emptyset} = \emptyset$.
\end{proof}

\begin{corollary}
	Tout espace métrique complet n'est pas de première catégorie.
\end{corollary}

\begin{proof}
	En effet, par $\ref{theorem_baire_complete_space}$, c'est un espace de
	Baire, et par le corollaire précédent, on a qu'il n'est pas de première catégorie.
\end{proof}

\begin{definition}
	Soit $(X, \tau)$ un espace topologique, et $A \subseteq X$.
	$A$ est \textbf{résiduel} s'il est le complémentaire d'un ensemble de
	première catégorie, ie $A^{c} \in \mathcal{C}_{1}$.
\end{definition}

\begin{proposition}
	\label{proposition_residual_equiv_baire}
	Soit $(X, \tau)$ un espace topologique.
	$(X, \tau)$ est de Baire ssi tout ensemble résiduel est dense, ie $\forall A
	\subseteq X$, $A^{c} \in \mathcal{C}_{1} \implies \adh{A} = X$.
\end{proposition}

\begin{proof}
	Reformulation de \ref{proposition_first_set_category_equiv_baire}. En effet,
	on a justement définie résiduel comme le complémentaire d'un ensemble de
	première catégorie.
\end{proof}

\begin{proposition}
	Dans tout espace topologique, tout fermé privé de son intérieur est
	d'intérieur vide.
\end{proposition}

\begin{proof}
	
\end{proof}

 \begin{proposition}
	Soit $(X, \tau)$ un espace de Baire, et \GSsequence{F}{n}{\naturel} une
	suite de fermés tel que $X = \displaystyle\cup_{n \in \naturel}F_{n}$.Alors
	$\displaystyle\adh{\cup_{n \in \naturel}\int{F_{n}}} = X$.
\end{proposition}

\begin{proof}
	
\end{proof}
