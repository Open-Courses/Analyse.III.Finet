\chapter{Ensemble de première catégorie d'un espace topologique}

\begin{definition} [Ensemble maigre ou nul part dense]
	Soit $(X, \tau)$ un espace topologique, et $A \subset X$. On dit que $A$ est
	\textbf{nul part dense} ou \textbf{maigre} si $\interior{\adh{A}} =
	\emptyset$. On note $\mathcal{C}_{npd}$ la classe des ensembles nul part
	denses.
\end{definition}

\begin{exemple}
	Si nous prenons l'espace topologique $(\real, \abs{.})$,
	\begin{enumerate}
		\item $\emptyset$, $\naturel$, $\integer$
		\item tous les ensembles finis.
	\end{enumerate}
	sont des ensembles nul part denses.
\end{exemple}

\begin{proposition}
	Soit $(X, \tau)$ un espace topologique.

	La classe des ensembles nul part denses est \textbf{héréditaire},
	c'est-à-dire
	\begin{equation}
		\forall A \in \mathcal{C}_{npd}, \forall B \subseteq A,
		B \in \mathcal{C}_{npd}
	\end{equation}
\end{proposition}

\ifdefined\outputproof
\begin{proof}
	\begin{align}
		B \subseteq A & \implies \adh{B} \subseteq \adh{A} \\
		& \implies \interior{\adh{B}} \subseteq \interior{\adh{A}}
	\end{align}
	et $\interior{\adh{A}}= \emptyset$ par
	hypothèse sur $A$. Donc $\interior{\adh{B}} = \emptyset$, et c'est la définition
	d'être nul part dense.
\end{proof}
\fi

\begin{proposition}
	Soit $(X, \tau)$ un espace topologique.

	$\mathcal{C}_{npd}$ est \textbf{stable par fermeture}, c'est-à-dire que
	$\forall A \in \mathcal{C}_{npd}$, $\adh{A} \in \mathcal{C}_{npd}$.
\end{proposition}

\ifdefined\outputproof
\begin{proof}
	$\adh{\adh{A}} = \adh{A}$, et donc, comme $A \in \mathcal{C}_{npd}$,
	$\interior{\adh{\adh{A}}} = \interior{\adh{A}} = \emptyset$
\end{proof}
\fi

\begin{proposition}
	Soit $(X, \tau)$ un espace topologique.

	$\mathcal{C}_{npd}$ est stable par union finie.
\end{proposition}

\ifdefined\outputproof
\begin{proof}
	Il suffit de montrer pour $n = 2$, le cas $n > 2$ se montre par récurrence.
	Soit $A$ et $B$ nul part dense. Il faut montrer que $A \union B$ est nul
	part dense, c'est-à-dire que $\interior{\adh{A \union B}} = \emptyset$. (A
	finir).
\end{proof}
\fi

\begin{definition} [Ensemble de première et deuxième catégorie]
	Soit $(X, \tau)$ un espace topologique.

	Un ensemble est de \textbf{première catégorie} s'il est une union
	dénombrable d'ensembles nul part denses dans $(X, \tau)$. Nous notons $\mathcal{C}_{1}$ la
	classe des ensembles de première catégorie.

	Sinon, on dit qu'il est \textbf{de deuxième catégorie}, et on
	note $\mathcal{C}_{2}$ la classe des ensembles de deuxième catégorie.
\end{definition}

Nous avons par exemple $\rational$ qui est de première catégorie dans $(\real,
\abs{.})$.

\begin{proposition}
	$\mathcal{C}_{1}$ est héréditaire et stable par union
	dénombrable.
\end{proposition}

\ifdefined\outputproof
\begin{proof}
	$\mathcal{C}_{1}$ stable par union dénombrable:

	Soit $A_{i} = \union_{n \in \naturel} A_{i, n}$ une famille dénombrable tel
	que chaque $A_{i, n}$ est nul part dense dans $(X, \tau)$. On a bien que
	$\union_{i \in \naturel,
	n \in \naturel} A_{i, n}$ est une union dénombrable (union sur
	$\naturel^{2}$ qui est dénombrable), et chaque $A_{i, n}$
	est nul part dense dans $(X, \tau)$ par hypothèse.

	$\mathcal{C}_{1}$ héréditaire:

	Soit $A = \union_{n \in \naturel} A_{n}$ un ensemble de première catégorie
	et soit $B \subseteq A$.

	Posons $B_{n} = A_{n} \inter B$. On a $B_{n} \subseteq A_{n}$ où $A_{n}$ est
	nul part dense dans $(X, \tau)$. Comme la classe des ensembles nul part
	dense est
	héréditaire, $B_{n}$ est nul part dense dans $(X, \tau)$.
	De plus, $\union_{n \in \naturel} B_{n} = \union_{n \in \naturel} (A_{n}
	\inter B) = \union_{n \in \naturel} A_{n} \inter B = A \inter B = B$.
\end{proof}
\fi

\begin{remarque}
	$\mathcal{C}_{1}$ n'est pas stable par fermeture.
	% Donner un exemple
\end{remarque}

\begin{definition} [$G_{\delta}$ et $F_{\sigma}$]
	Soit $(X, \tau)$ un espace topologique, et $A \subseteq X$.
	Alors $A$ est \textbf{un $G_{\delta}$ (resp. $F_{\sigma}$)} s'il est intersection
	dénombrable d'ouverts (resp. union dénombrable de fermés).
\end{definition}

\begin{definition} [Proposition vraie presque partout]
	Une proposition est \textbf{vraie presque partout} si elle est vraie sauf
	peut-être sur un ensemble de première catégorie.
\end{definition}

\section{Applications dans les espaces de Baire}

Remarquons d'abord que la propriété de Baire fermé
(\ref{property_baire_closed}) peut être écrit en toute union dénombrable
de fermés nul part denses est d'intérieur vide.

\begin{proposition}
	\label{proposition_first_set_category_equiv_baire}
	Soit $(X, \tau)$ est un espace topologique. Les assertions suivantes sont
	équivalentes.
	\begin{enumerate}
		\item $(X, \tau)$ est de Baire
		\item Le complémentaire de tout ensemble de première catégorie est dense
			dans $(X, \tau)$, c'est-à-dire pour tout $A \in \mathcal{C}_{1}$, $\adh{A^{c}} = X$.
	\end{enumerate}
\end{proposition}

\ifdefined\outputproof
\begin{proof}
	$(\Rightarrow)$
	Soit $S$ un ensemble de première catégorie, ie $S = \union_{n \in \naturel}
	S_{n}$ où $S_{n}$ est nul part dense. Il faut montrer que $\comp{S} =
	\inter_{n \in \naturel} \comp{(S_{n})}$ est dense dans $X$.

	(notation lourde)

	$(\Leftarrow)$
	Soit $\GSsequence{F}{n}{\naturel}$ une suite dénombrable de fermé tel que
	$\interior{F_{n}} = \emptyset$.
	Il faut montrer que $F = \union_{n \in \naturel} F_{n} = \emptyset$
	(\ref{property_baire_closed}).
	On a $F$ de première catégorie car chaque $F_{n}$ est nul part dense, et
	donc par hypothèse, $\adh{\comp{F}} = X$, c'est-à-dire $\interior{F} =
	\emptyset$ (en passant au complémentaire).
\end{proof}
\fi

\begin{corollary}
	Si $(X, \tau)$ est un ensemble de Baire, alors $X \notin \mathcal{C}_{1}$.
\end{corollary}

\ifdefined\outputproof
\begin{proof}
	Si $X$ est de première catégorie, alors $\emptyset = \comp{X}$ serait dense
	dans $X$. Or $\adh{\emptyset} = \emptyset$.
\end{proof}
\fi

\begin{corollary}
	Tout espace métrique complet n'est pas de première catégorie.
\end{corollary}

\ifdefined\outputproof
\begin{proof}
	En effet, par $\ref{theorem_baire_complete_space}$, c'est un espace de
	Baire, et par le corollaire précédent, on a qu'il n'est pas de première catégorie.
\end{proof}
\fi

\begin{definition} [Résiduel]
	Soit $(X, \tau)$ un espace topologique, et $A \subseteq X$.
	$A$ est \textbf{résiduel} s'il est le complémentaire d'un ensemble de
	première catégorie, ie $A^{c} \in \mathcal{C}_{1}$.
\end{definition}

\begin{proposition}
	\label{proposition_residual_equiv_baire}
	Soit $(X, \tau)$ un espace topologique. Alors les assertions suivantes sont
	équivalentes.
	\begin{enumerate}
		\item $(X, \tau)$ est de Baire.
		\item Tout ensemble résiduel est dense, c'est-à-dire pour tout
			sous-ensemble $A$ de $X$, si $A^{c} \in \mathcal{C}_{1}$ alors
			$\adh{A} = X$.
	\end{enumerate}
\end{proposition}

\ifdefined\outputproof
\begin{proof}
	Reformulation de \ref{proposition_first_set_category_equiv_baire}. En effet,
	on a justement définie résiduel comme le complémentaire d'un ensemble de
	première catégorie.
\end{proof}
\fi

\begin{proposition}
	Dans tout espace topologique, tout fermé privé de son intérieur est
	d'intérieur vide.
\end{proposition}

\ifdefined\outputproof
\begin{proof}

\end{proof}
\fi

\begin{proposition}
	Soit $(X, \tau)$ un espace de Baire et soit $E$ un sous-ensemble de $X$.

	Alors les assertions suivantes sont équivalentes.
	\begin{enumerate}
		\item $E$ est un ensemble résiduel.
		\item $E$ contient un sous-ensemble $G_{\delta}$ dense dans $(X, \tau)$.
	\end{enumerate}
\end{proposition}

\ifdefined\outputproof
\begin{proof}

\end{proof}
\fi

 \begin{proposition}
	Soit $(X, \tau)$ un espace de Baire, et $\GSsequence{F}{n}{\naturel}$ une
	suite de fermés tel que $X = \displaystyle\cup_{n \in \naturel}F_{n}$.

	Alors
	\begin{equation}
		\displaystyle \adh{\cup_{n \in \naturel}\interior{F_{n}}} = X
	\end{equation}
\end{proposition}

\ifdefined\outputproof
\begin{proof}

\end{proof}
\fi

\section{Applications à l'analyse fonctionnelle}

\subsection{Fonctions séparément continues}

\begin{theorem}
	\label{first_category_set_theorem_continued_function}
	Soit $\GSsequence{f}{n}{\naturel}$ tel que $f_{n}$ continue, et $f_{n} :
	(X, \tau) \rightarrow (\real, |.|)$.
	On pose $f(x) = \lim f_{n}(x)$ pour $x \in X$. Alors, $f$ est continue
	presque partout au sens de Baire.
\end{theorem}

\ifdefined\outputproof
\begin{proof}

\end{proof}
\fi

\begin{definition} [Séparément continue]
	$f$ est \textbf{séparément continue} si elle est continue en chacune de ses
	variables.
\end{definition}

\begin{proposition}
	Toute fonction séparément continue est limite simple de fonction continues.
\end{proposition}

\ifdefined\outputproof
\begin{proof}

\end{proof}
\fi

\begin{corollary}
	Soit $f : \real^{2} \rightarrow \real$ séparément continue (ie continue en
	chaque variable). Alors $f$ est presque partout continue au sens de Baire.
\end{corollary}

\ifdefined\outputproof
\begin{proof}
	On utilise la proposition précédente et le théorème
	\ref{first_category_set_theorem_continued_function}.
\end{proof}
\fi

\begin{corollary}
	La dérivée de toute fonction de $\real$ dans $\real$ dérivable est presque
	partout continue au sens de Baire.
\end{corollary}

\ifdefined\outputproof
\begin{proof}

\end{proof}
\fi

\subsection{Fonctions nulle part dérivables}

\begin{theorem}
	Presque toute fonction de $\mathcal{C}[0, 1]$ est nul part dérivable.
\end{theorem}

\ifdefined\outputproof
\begin{proof}

\end{proof}
\fi
