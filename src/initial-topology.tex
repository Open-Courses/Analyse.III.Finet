\chapter{Topologie initiale, cas particuliers et applications}

\section{Topologie initiale}

Prenons une famille
\GSsequence{E}{i}{I} d'espace topologique, où la topologie sur $E_{i}$ est notée
$\tau_{i}$, ainsi qu'une famille de fonctions \GSsequence{f}{i}{I} tel que
\GSfunction{$f_{i}$}{E}{$E_{i}$} où E est un ensemble.

Nous allons construire une topologie sur E, appelée \textbf{topologie initiale}
et noté $\tau$, telle qu'elle rend les applications $f_{i}$ continues. Il faut pour cela que
pour chaque $i \in I$, et pour chaque ouvert $O_{i} \in \tau_{i}$,
$f_{i}^{-1}(O_{i}) \in \tau$.

Nous posons une condition supplémentaire sur $\tau$ qui est qu'elle est la
topologie la moins fine qui rend les $f_{i}$ continues, c'est-à-dire que si on
prend une autre topologie $\tau^{*}$ sur E qui rend les $f_{i}$ continues, chaque
ouvert de $\tau$ doit être dans $\tau^{*}$.

Nous définissons $\tau$ comme %compléter%.
C'est bien une topologie vu la définition.

\begin{exemple}
	La topologie produit est un cas particulier de topologie initiale où $E =
	\displaystyle \prod_{i \in I} E_{i}$, et où les $f_{i}$ sont les projections.
\end{exemple}

\section{Topologie faible et préfaible}

--- Construire topologie faible et préfaible
\subsection{Applications}

On a construit précédemment sur $\GSdual{E}$ une topologie, qui est la topologie
préfaible, notée \GSpreweakTopo{E}.

Nous avons vu dans le chapitre sur les espaces de Banach, que dans un espace
vectoriel normé, la boule unité est compacte ssi l'espace est de dimension
finie. Dans notre cas, $E^{*}$ n'est pas de dimension finie si $E$ n'est pas de
dimension finie, donc la boule unité n'est pas compacte.

Or, la compacité joue un rôle important en analyse, et c'est là qu'intervient la topologie préfaible. On a le théorème suivant.

\begin{theorem}
	Soit l'espace topologique $(E,$ \GSpreweakTopo{E}$)$, alors la boule unité
	est compacte.
\end{theorem}

\begin{proof}
	
\end{proof}
