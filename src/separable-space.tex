\section{Espace séparable}

\begin{definition}
	Soit E un espace topologique.
	E est dit séparable s'il existe un sous-ensemble D tel que D est dense dans
	E et dénombrable.
\end{definition}

\begin{proposition}
	Soit $D \subseteq E$.
	D est dense dans E ssi D coupe tout ouvert de E.
\end{proposition}

\begin{proof}

\end{proof}

\begin{proposition}
	Prenons un espace topologique séparable X et une famille d'ouvert indicé par
	I \GSsequence{O}{i}{I}. Alors I est au plus dénombrable.
\end{proposition}

\begin{proof}
	
\end{proof}

\subsection{Séparabilité dans les espaces métriques}

Donnons quelques équivalences plus faciles à utiliser quand nous sommes dans un
espace métrique.

\begin{proposition}
	Soit $(E, d_{E})$ un espace métrique, LASSE :
	\begin{enumerate}
		\item $\forall \epsilon > 0$, le recouvrement $(B(x, \epsilon))_{x \in
			E}$ admet un sous-recouvrement \textbf{dénombrable}.
		\item E est séparable.
		\item $\exists$ \GSsequence{O}{n}{\naturel}, $O_{n}$ ouvert, tel que
			tout ouvert de E s'écrit comme une union de $O_{n}$.
	\end{enumerate}
\end{proposition}

\begin{proof}
	
\end{proof}

\begin{proposition}
	Si $(E, d_{E})$ est séparable, alors toute partie de E est séparable.
\end{proposition}

\begin{remarque}
	Cette proposition n'est pas vrai dans un espace simplement topologique.
	%Donner un exemple
\end{remarque}

\subsection{Séparabilité dans les espaces vectoriels normés}

Comme nous l'avons fait pour les espaces métriques, nous allons donner des
équivalences plus faciles à utiliser pour les espaces vectoriel normés.

\begin{proposition}
	Soit \GSnormedSpace{E}{\GSnorme{.}} un espace métrique, LASSE :
	\begin{enumerate}
		\item $E$ est séparable.
		\item $B(E)$ est séparable
		\item $S(E)$ est séparable
	\end{enumerate}
\end{proposition}

\begin{exemple}
	Prenons $c_{0}(\real)$, l'ensemble des suites à coefficients dans $\real$
	qui convergent vers 0. On a que $c_{0}$ est séparable.
\end{exemple}

\begin{exemple}
	Prenons $l^{1}(\real)$, l'ensemble des suites à coefficients dans $\real$
	tel que \GSnormeDef{x}{1} = \GSsum{n}{0}{\infty}{$x_{n}$} < $\infty$ est
	séparable.
\end{exemple}

\begin{exemple}
	$l^{\infty}(\real)$, l'ensemble des suites bornées à coefficients dans
	$\real$ n'est pas séparable.
\end{exemple}

%Théorème des Stone-Weierstrass ??
