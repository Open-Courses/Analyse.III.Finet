\chapter{Espace séparable}

Pour rappel, un ensemble $D$ est \textbf{dense} dans $E$ si pour tout $x_{0} \in E$, $D$
coupe tout voisinage de $x_{0}$.

\section{Séparabilité dans un espace topologique}

\begin{definition}
	Soit $E$ un espace topologique.
	On dit que $E$ est \textbf{séparable} s'il existe un sous-ensemble $D$ de
	$E$ tel que
	\begin{enumerate}
		\item $D$ est dense dans $E$.
		\item $D$ est dénombrable.
	\end{enumerate}
\end{definition}

\begin{proposition}
	Soit $D \subseteq E$.
	Alors les assertions suivantes sont équivalentes.
	\begin{enumerate}
		\item $D$ est dense dans $E$.
		\item $D$ coupe tout ouvert non vide de $E$.
	\end{enumerate}
\end{proposition}

\ifdefined\outputproof
\begin{proof}
	Reformulation de la définition en terme d'ouvert. En effet, un ouvert est un
	voisinage de tous ses points.
\end{proof}
\fi

La proposition suivante caractérise les topologies des espaces séparables. Elle
nous dit que si un espace est séparable, alors il a au plus un nombre
dénombrable d'ouverts disjoints.

\begin{proposition}
	Soit $(X, \tau)$ un espace topologique séparable et soit $\GSsequence{O}{i}{I}$ une famille d'ouvert
	2 à 2 disjoints.

	Alors $I$ est au plus dénombrable.
\end{proposition}

\ifdefined\outputproof
\begin{proof}
	Soit $D$ un sous-ensemble dénombrable de $X$ et dense dans $(X, \tau)$. On sait que
	\begin{equation}
		\forall i \in I, D \inter O_{i} \neq \emptyset
	\end{equation}
	car $D$ est dense dans $(X, \tau)$ et $O_{i}$ est un ouvert.
	Notons $x_{i}$ l'élément qui se trouve dans $D \inter O_{i}$.

	Construisons
	\begin{equation}
		\GSfunction{f}{I}{D} : i \rightarrow x_{i}
	\end{equation}

	Nous avons que $f$ est injective.
	En effet, pour $i \neq j$, si $x_{i} = x_{j}$ avec $x_{j} \in O_{j}$ et
	$x_{i} \in O_{i}$, alors $x_{j} \in O_{i}$ et donc $O_{i}$ et $O_{j}$ ne sont pas
	disjoints.

	Donc $Card(I) \leq Card(D)$. Donc $I$ est au plus dénombrable car $D$ est
	dénombrable.
\end{proof}
\fi

\section{Séparabilité dans les espaces métriques}

Donnons quelques équivalences plus faciles à utiliser quand nous sommes dans un
espace métrique.

\begin{proposition}
	Soit $(E, d_{E})$ un espace métrique. Alors les assertions suivantes sont
	équivalentes.
	\begin{enumerate}
		\item pour tout $\epsilon > 0$, le recouvrement $(B(x, \epsilon))_{x \in
			E}$ admet un sous-recouvrement \textbf{dénombrable}.
		\item E est séparable.
		\item il existe une suite $\GSsequence{O}{n}{\naturel}$ d'ouverts de $(E,
			d_{E})$, tel que tout ouvert de $E$ s'écrit comme une union de $O_{n}$.
	\end{enumerate}
\end{proposition}

\ifdefined\outputproof
\begin{proof}

\end{proof}
\fi

\begin{proposition}
	Si $(E, d_{E})$ est séparable, alors toute partie de E est séparable.
\end{proposition}

\ifdefined\outputproof
\begin{proof}

\end{proof}
\fi

\begin{remarque}
	Cette proposition n'est pas vraie dans un espace simplement topologique.
	%Donner un exemple
\end{remarque}

\section{Séparabilité dans les espaces vectoriels normés}

Comme nous l'avons fait pour les espaces métriques, nous allons donner des
équivalences plus faciles à utiliser pour les espaces vectoriels normés.

Remarquons qu'un espace vectoriel normé est en particulier un espace
métrique, donc les équivalences de la section précédente restent vraies.

\begin{proposition}
	Soit $\GSnormedSpace{E}{\GSnorme{.}}$ un espace vectoriel normé. Alors les
	assertions suivantes sont équivalentes.
	\begin{enumerate}
		\item $E$ est séparable.
		\item $B(E)$ est séparable
		\item $S(E)$ est séparable
	\end{enumerate}
\end{proposition}

\ifdefined\outputproof
\begin{proof}

\end{proof}
\fi

\begin{proposition}
	L'ensemble des suites à coefficients dans $\real$ qui convergent vers 0,noté
	$c_{0}(\real)$, est séparable.
\end{proposition}

\ifdefined\outputproof
\begin{proof}

\end{proof}
\fi

\begin{proposition}
	L'ensemble des suites $\GSsequence{x}{n}{\naturel}$ à coefficients dans $\real$ tel que
	\begin{equation}
		\GSnormeDef{x}{1} = \GSsum{n}{0}{\infty}{\abs{x_{n}}} < \infty
	\end{equation}
	noté $l^{1}(\real)$, est séparable.
\end{proposition}

\ifdefined\outputproof
\begin{proof}

\end{proof}
\fi

\begin{proposition}
	L'ensemble des suites bornées à coefficients dans $\real$, noté
	$l^{\infty}(\real)$, n'est pas séparable.
\end{proposition}

\ifdefined\outputproof
\begin{proof}

\end{proof}
\fi

\begin{question}
	On a vu à travers les exemples que $c_{0}$, et $l^{1}$ sont séparables,
	tandis que $l^{\infty}$ non. On a la chaîne d'inclusion $c_{0} \subseteq
	l^{1} \subseteq l^{\infty}$.
	On pourrait se poser comme question si toute chaîne d'espace séparable
	est bornée par un ensemble non séparable, en d'autres termes est-ce que tout
	ensemble séparable est contenu dans un ensemble non séparable? (En
	utilisant la topologie induite).
\end{question}
%Théorème de Stone-Weierstrass ??
