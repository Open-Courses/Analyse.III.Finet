\chapter{Espace séparable}

Pour rappel, un ensemble $D$ est dense dans $E$ si pour tout $x_{0} \in E$, $D$
coupe tout voisinage de $x_{0}$.

\section{Séparabilité dans un espace topologique}

\begin{definition}
	Soit $E$ un espace topologique.
	$E$ est dit séparable s'il existe un sous-ensemble $D$ tel que $D$ est dense
	dans $E$ et dénombrable.
\end{definition}

\begin{proposition}
	Soit $D \subseteq E$.
	$D$ est dense dans $E$ ssi $D$ coupe tout ouvert non vide de $E$.
\end{proposition}

\begin{proof}
	Reformulation de la définition en terme d'ouvert. En effet, un ouvert est un
	voisinage de tous ses points.
\end{proof}

La proposition suivante caractérise les topologies des espaces séparables. Elle
nous dit que si un espace est séparable, alors il a au plus un nombre
dénombrable d'ouverts disjoints.

\begin{proposition}
	Prenons un espace topologique séparable $X$ et une famille d'ouvert
	$\GSsequence{O}{i}{I}$ 2 à 2 disjoints. Alors $I$ est au plus dénombrable.
\end{proposition}

\begin{proof}
	Soit $D$ dense dans $E$ et dénombrable. On sait que $\forall i \in I$, $D$
	coupe $O_{i}$. Notons $x_{i}$ l'élément qui se trouve dans $D \bigcap
	O_{i}$.

	On construit $\GSfunction{f}{I}{D}$ : $i \rightarrow x_{i}$. On a $f$ qui
	est injective car si $x_{i} = x_{j}$ ($i \neq j$), avec $x_{j} \in O_{j}$ et
	$x_{i} \in O_{i}$, $x_{j} \in O_{i}$ et donc $O_{i}$ et $O_{j}$ ne sont pas
	disjoints.

	Donc $Card(I) \leq Card(D)$. Donc $I$ au plus dénombrable car $D$ dénombrable.
\end{proof}

\section{Séparabilité dans les espaces métriques}

Donnons quelques équivalences plus faciles à utiliser quand nous sommes dans un
espace métrique.

\begin{proposition}
	Soit $(E, d_{E})$ un espace métrique, LASSE :
	\begin{enumerate}
		\item $\forall \epsilon > 0$, le recouvrement $(B(x, \epsilon))_{x \in
			E}$ admet un sous-recouvrement \textbf{dénombrable}.
		\item E est séparable.
		\item $\exists \GSsequence{O}{n}{\naturel}$, $O_{n}$ ouvert, tel que
			tout ouvert de E s'écrit comme une union de $O_{n}$.
	\end{enumerate}
\end{proposition}

\begin{proof}

\end{proof}

\begin{proposition}
	Si $(E, d_{E})$ est séparable, alors toute partie de E est séparable.
\end{proposition}

\begin{proof}

\end{proof}

\begin{remarque}
	Cette proposition n'est pas vraie dans un espace simplement topologique.
	%Donner un exemple
\end{remarque}

\section{Séparabilité dans les espaces vectoriels normés}

Comme nous l'avons fait pour les espaces métriques, nous allons donner des
équivalences plus faciles à utiliser pour les espaces vectoriel normés. Il ne
faut pas oublier qu'un espace vectoriel normé est en particulier un espace
métrique, donc les équivalences de la section précédente restent vraies.

\begin{proposition}
	Soit $\GSnormedSpace{E}{\GSnorme{.}}$ un espace vectoriel normé, LASSE :
	\begin{enumerate}
		\item $E$ est séparable.
		\item $B(E)$ est séparable
		\item $S(E)$ est séparable
	\end{enumerate}
\end{proposition}

\begin{proof}

\end{proof}

\begin{proposition}
	$c_{0}(\real)$, l'ensemble des suites à coefficients dans $\real$
	qui convergent vers 0, est séparable.
\end{proposition}

\begin{proof}

\end{proof}

\begin{proposition}
	$l^{1}(\real)$, l'ensemble des suites à coefficients dans $\real$
	tel que $\GSnormeDef{x}{1} = \GSsum{n}{0}{\infty}{x_{n}} < \infty$, est
	séparable.
\end{proposition}

\begin{proof}

\end{proof}

\begin{proposition}
	$l^{\infty}(\real)$, l'ensemble des suites bornées à coefficients dans
	$\real$ n'est pas séparable.
\end{proposition}

\begin{proof}

\end{proof}

\begin{question}
	On a vu à travers les exemples que $c_{0}$, et $l^{1}$ sont séparables,
	tandis que $l^{\infty}$ non. On a la chaîne d'inclusion $c_{0} \subseteq
	l^{1} \subseteq l^{\infty}$.
	On pourrait se poser comme question si toute chaîne d'espace séparable
	est bornée par un ensemble non séparable, en d'autres termes est-ce que tout
	ensemble séparable est contenu dans un ensemble non séparable? (En
	utilisant la topologie induite).
\end{question}
%Théorème de Stone-Weierstrass ??
