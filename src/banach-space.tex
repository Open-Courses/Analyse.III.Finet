\section{Espace de Banach}

\begin{definition} [Espace de Banach]
	Soit \GSnormedSpace{E}{\GSnorme{.}} un espace vectoriel normé. On dit que E
	est un espace de Banach si E est complet pour la norme \GSnorme{.},
	c'est-à-dire que toutes suites de cauchy convergent.
\end{definition}

Prenons $\real$ comme corps de base, alors tout espace vectoriel de dimension n
est complet.

On peut généraliser cet exemple si on prend un corps de base complet, et
n'importe quel espace vectoriel de dimension fini sur ce corps, comme $\complex$

\begin{proposition}
	Soit K un corps, et E un espace vectoriel de dimension finie sur K. On a que
	K est complet ssi E est un espace de Banach.
\end{proposition}

\begin{question}
	Existe-t-il un corps complet dont tout espace vectoriel sur ce corps est un
	espace de Banach ?

	De même, existe-t-il un corps qui n'est pas complet et qui n'admet aucun
	espace vectoriel complet.
\end{question}

\begin{exercice}
	Montrez que tout corps fini est complet. En déduire que tout espace
	vectoriel de dimension fini sur un corps fini est de Banach.
\end{exercice}

\begin{exercice}
	On sait que $\real$ est complet en tant que corps, et que $\real$ est
	peut-être vu comme un $\rational$-espace vectoriel. Déduire que $\real$
	n'est pas un $\rational$-espace vectoriel de dimension finie.
\end{exercice}

Le cas des espaces vectoriels de dimensions finie n'étant pas très intéressant
pour la propriété de Banach, nous nous intéresserons surtout aux espaces
vectoriels de dimension infinie.
