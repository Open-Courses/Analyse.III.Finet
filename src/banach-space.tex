h\chapter{Espace de Banach}

\section{Rappels sur les espaces vectoriels normés}

\begin{definition} [Espace de Banach]
	Soit $\GSnormedSpace{E}{\GSnorme{.}}$ un espace vectoriel normé. On dit que E
	est \textbf{un espace de Banach} si E est complet pour la norme $\GSnorme{.}$,
	c'est-à-dire que toutes suites de cauchy convergent pour la norme
	$\GSnorme{.}$.
\end{definition}

Prenons $\real$ comme corps de base, alors tout espace vectoriel de dimension n
est complet.

On peut généraliser cet exemple si on prend un corps de base complet, et
n'importe quel espace vectoriel de dimension finie sur ce corps.

\begin{theorem} [Riesz]
\label{theorem_riesz_compact}
	$E$ est de dimension finie ssi la boule unité fermée est compacte.
\end{theorem}

\ifdefined\outputproof
\begin{proof}

\end{proof}
\fi

\begin{proposition}
\label{proposition_dual_dimension}
	Soit $\GSnormedSpace{E}{\GSnorme{.}}$ un espace vectoriel normé.
	Alors, les assertions suivantes sont équivalentes.
	\begin{enumerate}
		\item $E^{*}$ est de dimension finie.
		\item $E$ est de dimension finie.
	\end{enumerate}
\end{proposition}

\ifdefined\outputproof
\begin{proof}

\end{proof}
\fi

\begin{theorem}
\label{theorem_closed_vectorial_subspace}
	Soit $E$ un espace de Banach, et $F$ un sous espace vectoriel de $E$ de
	dimension finie.
	Alors $F$ est fermé.
\end{theorem}

\ifdefined\outputproof
\begin{proof}

\end{proof}
\fi

\begin{remarque}
	La demande que $E$ soit complet est important car sinon c'est faux. En
	effet, tout espace vectoriel de dimension finie sur $\rational$ n'est pas
	complet.
\end{remarque}

\section{Définition, propriétés et exercices}


\begin{proposition}
	Soit K un corps, et E un espace vectoriel de dimension finie sur K. Alors
	les assertions suivantes sont équivalentes.
	\begin{enumerate}
		\item $K$ est complet.
		\item $E$ est un espace de Banach.
	\end{enumerate}
\end{proposition}

\ifdefined\outputproof
\begin{proof}

\end{proof}
\fi

\begin{question}
	Existe-t-il un corps complet dont tout espace vectoriel sur ce corps est un
	espace de Banach?

	De même, existe-t-il un corps qui n'est pas complet et qui n'admet aucun
	espace vectoriel complet.
\end{question}

\begin{exercice}
	Montrez que tout corps fini est complet. En déduire que tout espace
	vectoriel de dimension finie sur un corps fini est de Banach. Qu'en est-il
	des espaces vectoriels de dimension infinie?
\end{exercice}

\begin{exercice}
	On sait que $\real$ est complet en tant que corps, et que $\real$ est
	peut-être vu comme un $\rational$-espace vectoriel. Déduire que $\real$
	n'est pas un $\rational$-espace vectoriel de dimension finie.
\end{exercice}

\begin{proposition}
	Soit $\GSnormedSpace{E}{\GSnorme{.}}$ un espace vectoriel normé.
	Alors les assertions suivantes sont équivalentes.
	\begin{enumerate}
		\item $E$ est un espace de Banach.
		\item toute série absolument convergente est convergente.
	\end{enumerate}
\end{proposition}

\ifdefined\outputproof
\begin{proof}

\end{proof}
\fi

\begin{exemple}
	$c_{00}$ n'est pas complet.
\end{exemple}

