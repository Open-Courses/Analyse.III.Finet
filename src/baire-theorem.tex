\section{Théorème de Baire}

Énonçons d'abord une propriété, qu'on appelle propriété de Baire.

\begin{propriete} [Propriété de Baire ouverts]
	Pour toutes suites dénombrables d'ouverts denses dans E, on a $\displaystyle
	\bigcap_{n = 0}^{\infty} O_{n}$ qui est dense dans E.
\end{propriete}

On a alors une proposition équivalente à la propriété de Baire en terme de
fermés.

\begin{propriete} [Propriété de Baire fermés]
	Pour toutes suites dénombrables de fermé d'intérieur vide, on a $E \neq
	\displaystyle \bigcup_{n = 0}^{\infty} F_{n}$.
\end{propriete}

\begin{proof}
	
\end{proof}

Nous avons alors un corollaire (en supposant que la propriété de Baire est
vraie) :

\begin{corollary}
	Soit une suite \GSsequence{F}{n}{\naturel} de fermés tel que $\displaystyle
	\bigcup_{n = 0}^{\infty}{F_{n}} = E$, alors il existe un $F_{n_{0}}$
	d'intérieur non vide.
\end{corollary}

\begin{proof}
	
\end{proof}

\begin{definition} [Espace de Baire]
	Si un espace topologie E a la propriété de Baire, on dit que c'est un espace
	de Baire.
\end{definition}

La question à se poser est : est-ce qu'il existe des espaces de Baire. La
réponse est oui, et la proposition suivante nous en donne une partie.

\begin{proposition}
	Tout espace métrique complet est un espace de Baire.
\end{proposition}

\begin{proof}
	
\end{proof}

\begin{remarque}
	Si on considère les suites d'ouverts non dénombrables pour la propriété de
	Baire, nous n'avons pas la proposition précédente (exemple avec $\real$).
	%Donner le contre-exemple
\end{remarque}

On a alors comme conséquence un proposition très intéressante sur les bases
algébriques des espaces vectoriels normés E. Pour rappel, une base algébrique
est un ensemble d'éléments B de E linéairement indépendants tel que tout élément
de E s'écrit comme une combinaire linéaire finie des éléments de B.

\begin{proposition}
	Soit E un espace de Banach, alors toute base algébrique est soit finie, soit
	non-dénombrable.
\end{proposition}

On a alors comme exemple d'application de la proposition que $\mathcal{C}[0, 1]$
et \GScontinueHomo{E}{F}, avec E ou F de dimension infinie, et F de Banach,
n'ont que des bases algébriques non dénombrables (car ce ne sont pas des espaces
vectoriels de dimension finie).

On a alors, grace à la contraposée, un corollaire important sur les espaces de
Banach :

\begin{corollary}
	Soit E un espace vectoriel normé qui a une base algébrique dénombrable.
	Alors E n'est pas complet (ou E n'est pas de Banach).
\end{corollary}
