\chapter{Théorème de Baire}
\label{chapter_baire_theorem}

Énonçons d'abord une propriété, qu'on appelle propriété de Baire.

\begin{propriete} [Propriété de Baire ouverts]
	\label{property_baire_open}
	Pour toutes suites dénombrables $\GSsequence{O}{n}{\naturel}$ d'ouverts denses dans E, on a $\displaystyle
	\bigcap_{n = 0}^{\infty} O_{n}$ qui est dense dans E.
\end{propriete}

On a alors une proposition équivalente à la propriété de Baire en terme de
fermés.

\begin{propriete} [Propriété de Baire fermés]
\label{property_baire_closed}
Pour toutes suites dénombrables $\GSsequence{F}{n}{\naturel}$ de fermés d'intérieur vide, on a
	
	\begin{equation}
		\interior{\bigcup_{n = 0}^{\infty} F_{n}} = \emptyset
	\end{equation}
\end{propriete}

\ifdefined\outputproof
\begin{proof}
	Soit $\GSsequence{F}{n}{\naturel}$ une suite de fermés d'interieur vide et
	posons $O_{n} := \comp{F_{n}}$. Alors, pour tout $n \in \naturel$, on a
	$O_{n}$ qui est ouvert et dense car
	\begin{align}
		\interior{F_{n}} = \emptyset &\Leftrightarrow \adh{\comp{F_{n}}} = E \\
		& \Leftrightarrow \adh{O_{n}} = E
	\end{align}

	Posons maintenant $F = \bigcup_{n = 0}^{\infty} F_{n}$. On a $\comp{F} = \bigcap_{n = 0}^{\infty} O_{n}$.

	\begin{align}
		\interior{F} = \emptyset & \Leftrightarrow \adh{F^{c}} = E \\
		& \Leftrightarrow \adh{\bigcap_{n = 0}^{\infty} O_{n}} = E
	\end{align}
	Donc $F^{c}$ est ouvert et dense.
\end{proof}
\fi

Nous avons alors un corollaire (en supposant que la propriété de Baire est
vraie):

\begin{corollary}
\label{corollary_baire_int_non_empty}
	Soit une suite $\GSsequence{F}{n}{\naturel}$ de fermés dont l'intérieur de
	l'union est non vide, c'est-à-dire 
	\begin{equation}
		\interior{\left(\bigcup_{n = 0}^{\infty}{F_{n}}\right)} \neq \emptyset
	\end{equation}
	alors il existe un $n_{0} \in \naturel$ tel que $F_{n_{0}}$
	est d'intérieur non vide.
\end{corollary}

\ifdefined\outputproof
\begin{proof}
	Supposons que tous les $F_{n}$ soient d'intérieur vide, alors par la
	propriété de Baire fermé (~\ref{property_baire_closed}), on a
	\begin{equation}
		\interior{\left(\bigcup_{n = 0}^{\infty}{F_{n}} \right)} = \emptyset
	\end{equation}
	Or, par hypothèse, ce n'est pas vide.
\end{proof}
\fi

\begin{definition} [Espace de Baire]
\label{definition_baire_space}
	Si un espace topologie E a la propriété de Baire, on dit que c'est \textbf{un espace
	de Baire}.
\end{definition}

La question à se poser est: est-ce qu'il existe des espaces de Baire? La
réponse est oui, et le théorème (de Baire) nous en donne une partie.

\begin{theorem} [Théorème de Baire]
\label{theorem_baire_complete_space}
	Tout espace métrique complet est un espace de Baire.
\end{theorem}

\ifdefined\outputproof
\begin{proof}

\end{proof}
\fi

\begin{remarque}
	Si on considère les suites d'ouverts non dénombrables pour la propriété de
	Baire, nous n'avons pas la proposition précédente.
	En effet, si on prend, pour $\real$ qui est complet, tous les singletons,
	qui sont fermés et d'intérieur vide, recouvrent $\real$. Or, cela
	contredit la propriété de Baire fermé~\ref{property_baire_closed}.
\end{remarque}

On a alors comme conséquence une proposition très intéressante sur les bases
algébriques des espaces vectoriels normés
$\GSnormedSpace{E}{\GSnormeDef{.}{E}}$. Pour rappel, une base algébrique
est un ensemble d'éléments B de E linéairement indépendants tel que tout élément
de E s'écrit comme une combinaire linéaire finie des éléments de B.

\begin{proposition}
\label{proposition_basis_banach_space}
	Soit $\GSnormedSpace{E}{\GSnormeDef{.}{E}}$ un espace de Banach, alors toute base algébrique est soit finie, soit
	non-dénombrable.
\end{proposition}

\ifdefined\outputproof
\begin{proof}
	Soit $\GSnormedSpace{E}{\GSnormeDef{.}{E}}$ un espace de Banach.
	On a par~\ref{theorem_baire_complete_space} que c'est un espace de Baire.

	Si $E$ est de dimension finie, on a fini.
	Supposons maintenant qu'il soit de dimension infinie dénombrable, et prenons
	une base $B = \GSsequence{e}{n}{\naturel}$.

	On construit alors la suite $E_{n} = \spanspace{e_{1}, \ldots, e_{n}}$. Ce sont des
	sous espaces vectoriels de dimension finie, et donc fermés
	par~\ref{theorem_closed_vectorial_subspace}. De plus, on a
	clairement $E = \bigcup_{n \in \naturel} E_{n}$.

	-- Par~\ref{corollary_baire_int_non_empty}, il existe $n_{0} \in \naturel$ tel que $E_{n_{0}}$ est d'intérieur non
	vide.
	Comme $E_{n_{0}}$ est d'intérieur non vide, il existe une
	boule $B(x_{0}, r)$ de centre $x_{0} \in int({E_{n_{0}}})$ et de rayon $r > 0$ tel que
	$B(x_{0}, r) \subseteq int(E_{n_{0}})$. En particulier, $B(x_{0}, r)
	\subseteq E_{n_{0}}$.
	
	On a $B(x_{0}, r) = x_{0} + B(0, r) \subseteq
	E_{n_{0}}$

	Comme $E_{n_{0}}$ est un espace vectoriel, si on prend un
	élément $x_{0} + x \in E_{n_{0}}$ où $x \in B(0, r)$, alors $x \in E_{n_{0}}$.
	Donc $B(0, r) \subseteq E_{n_{0}}$.

	-- Montrons maintenant que $E_{n_{0}} = E$.

	Prenons un élément $x$ de $E$ non nul et posons
	\begin{equation}
		y = \frac{x}{||x||} \frac{r}{2}
	\end{equation}
	On a $y \in B(0, r)$ et par conséquent $y \in E_{n_{0}}$.
	
	Comme $E_{n_{0}}$ est un espace vectoriel et que 
	\begin{equation}
		x = \frac{2}{r} ||x|| y
	\end{equation}
	on a $x \in E_{n_{0}}$. D'où $E \subseteq E_{n_{0}}$.
	
	On a donc bien l'égalité vu que $E_{n_{0}}$ est un sous-espace vectoriel de
	$E$ par construction.

	On obtient alors une contradiction car $E_{n_{0}}$ est de dimension finie,
	alors que $E$ est de dimension infinie dénombrable.
\end{proof}
\fi

On a alors comme exemple d'application de la
proposition~\ref{proposition_basis_banach_space} que $\mathcal{C}[0, 1]$ et
$\GScontinueHomo{E}{F}$, avec $E$ ou $F$ de dimension infinie, et $F$ de Banach,
n'ont que des bases algébriques non dénombrables (car ce ne sont pas des espaces
vectoriels de dimension finie).

Donnons maintenant, grace à la contraposée, un corollaire important sur les espaces de
Banach.

\begin{corollary}
	Soit $\GSnormedSpace{E}{\GSnormeDef{.}{E}}$ un espace vectoriel normé qui a
	une base algébrique dénombrable.
	Alors $\GSnormedSpace{E}{\GSnormeDef{.}{E}}$ n'est pas un espace de Banach.
\end{corollary}

\ifdefined\outputproof
\begin{proof}
	Si $\GSnormedSpace{E}{\GSnormeDef{.}{E}}$ était un espace de Banach, il
	n'aurait pas de base dénombrable.
\end{proof}
\fi
