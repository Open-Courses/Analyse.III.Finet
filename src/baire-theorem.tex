\chapter{Théorème de Baire}

Énonçons d'abord une propriété, qu'on appelle propriété de Baire.

\begin{propriete} [Propriété de Baire ouverts]
\label{property_baire_open}
	Pour toutes suites dénombrables d'ouverts denses dans E, on a $\displaystyle
	\bigcap_{n = 0}^{\infty} O_{n}$ qui est dense dans E.
\end{propriete}

On a alors une proposition équivalente à la propriété de Baire en terme de
fermés.

\begin{propriete} [Propriété de Baire fermés]
\label{property_baire_closed}
	Pour toutes suites dénombrables de fermé d'intérieur vide, on a
	$\displaystyle \bigcup_{n = 0}^{\infty} F_{n} = \emptyset$.
\end{propriete}

\begin{proof}
	$int(F) = \emptyset \Leftrightarrow adh(F^{c}) = E$ où $F$ est fermé. Donc
	$F^{c}$ est ouvert et dense.
\end{proof}

Nous avons alors un corollaire (en supposant que la propriété de Baire est
vraie):

\begin{corollary}
\label{corollary_baire_int_non_empty}
	Soit une suite \GSsequence{F}{n}{\naturel} de fermés tel que $\displaystyle
	\bigcup_{n = 0}^{\infty}{F_{n}} = E$, alors il existe un $F_{n_{0}}$
	d'intérieur non vide.
\end{corollary}

\begin{proof}
	Supposons que tous les $F_{n}$ soient fermés, alors par la
	propriété de Baire fermé (~\ref{property_baire_closed}), on a que
	$\displaystyle \bigcup_{n = 0}^{\infty}{F_{n}} \neq E$. Or, par hypothèse,
	on a l'égalité.
\end{proof}

\begin{definition} [Espace de Baire]
\label{definition_baire_space}
	Si un espace topologie E a la propriété de Baire, on dit que c'est un espace
	de Baire.
\end{definition}

La question à se poser est: est-ce qu'il existe des espaces de Baire? La
réponse est oui, et le théorème (de Baire) nous en donne une partie.

\begin{theorem} [Théorème de Baire]
\label{theorem_baire_complete_space}
	Tout espace métrique complet est un espace de Baire.
\end{theorem}

\begin{proof}
	
\end{proof}

\begin{remarque}
	Si on considère les suites d'ouverts non dénombrables pour la propriété de
	Baire, nous n'avons pas la proposition précédente.
	En effet, si on prend, pour $\real$ qui est complet, tous les singletons,
	qui sont fermés et d'intérieur vide, ils recouvrent $\real$. Or, cela
	contredit la propriété de Baire fermé~\ref{property_baire_closed}.
\end{remarque}

On a alors comme conséquence une proposition très intéressante sur les bases
algébriques des espaces vectoriels normés E. Pour rappel, une base algébrique
est un ensemble d'éléments B de E linéairement indépendants tel que tout élément
de E s'écrit comme une combinaire linéaire finie des éléments de B.

\begin{proposition}
\label{proposition_basis_banach_space}
	Soit E un espace de Banach, alors toute base algébrique est soit finie, soit
	non-dénombrable.
\end{proposition}

\begin{proof}
	Soit $E$ un espace de Banach. On a donc
	par~\ref{theorem_baire_complete_space} que c'est un espace de Baire.

	Si $E$ est de dimension finie, on a fini.
	Supposons maintenant qu'il soit de dimension infinie dénombrable, et prenons
	une base $B = $ \GSsequence{e}{n}{\naturel}.

	On construit alors la suite $E_{n} = <e_{1}, \ldots, e_{n}>$. Ce sont des
	sous espaces vectoriels de dimension finie, et donc fermés
	par~\ref{theorem_closed_vectorial_subspace}. De plus, on a
	clairement $E = \bigcup_{n \in \naturel} E_{n}$.

	Par~\ref{corollary_baire_int_non_empty}, on a un $E_{n_{O}}$ d'intérieur non
	vide. Il contient dont une boule de centre $x_{0} \in int({E_{n_{0}}})$ et
	de rayon $r > 0$ ($B(x_{0}, r) \subseteq E_{n_{0}}$).
	On a $B(x_{O}, r) = x_{0} + B(0, r) \subseteq E_{n_{0}}$, et comme
	$E_{n_{0}}$ est un espace vectoriel, si on prend un élément $x_{0} + x \in
	E_{n_{0}}$ où $x \in B(0, r)$, $x \in E_{n_{0}}$, donc $B(0, r) \subseteq
	E_{n_{0}}$.

	Montrons maintenant que $E_{n_{0}} = E$. Si on prend un élément $x$ de $E$
	non nul, on a $y = \frac{x}{||x||} \frac{r}{2} \in B(0, r)$, donc $y \in
	E_{n_{0}}$. Comme $E_{n_{0}}$ est un espace vectoriel, $x = \frac{2}{r}
	||x|| y \in E_{n_{0}}$, d'où $E \subseteq E_{n_{0}}$. On a donc bien
	l'égalité vu que $E_{n_{0}}$ est un sev de $E$.

	On obtient alors une contradiction car $E_{n_{0}}$ est de dimension finie,
	alors que $E$ est de dimension infinie dénombrable.
\end{proof}

On a alors comme exemple d'application de la
proposition~\ref{proposition_basis_banach_space} que $\mathcal{C}[0, 1]$ et
\GScontinueHomo{E}{F}, avec E ou F de dimension infinie, et F de Banach,
n'ont que des bases algébriques non dénombrables (car ce ne sont pas des espaces
vectoriels de dimension finie).

On a alors, grace à la contraposée, un corollaire important sur les espaces de
Banach :

\begin{corollary}
	Soit E un espace vectoriel normé qui a une base algébrique dénombrable.
	Alors E n'est pas complet (ou E n'est pas de Banach).
\end{corollary}

\begin{proof}
	S'il était complet, il n'aurait pas de base dénombrable.
\end{proof}
