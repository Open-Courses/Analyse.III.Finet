\chapter{Le théorème d'Ascoli}

\begin{definition}
	Soit $(X, \tau)$ un espace topologique et $(Y, d_{Y})$ un espace métrique.
	Posons
	\begin{equation}
		F(X, Y) = \GSset{f : X \rightarrow Y}
	\end{equation}
	l'ensemble des fonctions de $X$ dans $Y$.

	Soient $A \subseteq F(X, Y)$ et $x_{0} \in X$.
	On dit que $A$ est \textbf{équicontinu en $x_{0}$} si pour tout $\epsilon >
	0$, il existe un voisinage ouvert $V_{x_{0}}$ de $x_{0}$ tel que pour toute
	fonction $f \in A$, $d_{Y}(f(x), f(x_{0})) \leq \epsilon$.

	On dit que $A$ est \textbf{équicontinu} si $A$ est équicontinu en tout point
	$x_{0}$ de $X$.
\end{definition}

On va munir $F(X, Y)$ d'une topologie dépendante de la topologie $\tau$ et de
la métrique $d_{Y}$.

\begin{definition} [Topologie de la convergence uniforme]

\end{definition}

\begin{exemple}
	\begin{enumerate}
		\item Soient $(X, d_{X})$ et $(Y, d_{Y})$ deux espaces métriques.
			L'ensemble des fonctions $k$-lipchitziennes pour $k > 0$ est
			équicontinu.
		\item Soient $(X, d_{X})$ et $(Y, d_{Y})$ deux espaces métriques. Soit
			$F : (X, d_{X}) \cartprod (X, d_{X}) \rightarrow (Y, d_{Y})$ une
			fonction uniformément continue et posons $A = \GSsetDef{f_{y} : X
		\rightarrow Y}{f_{y}(x) = f(x, y)}$. Alors $A$ est équicontinu.
	\end{enumerate}
\end{exemple}


\begin{proposition}
	Soient $(X, \tau)$ un espace topologique et $(Y, d_{Y})$ un espace métrique
	et soit $A \subseteq F(X, Y)$. Alors

	\begin{enumerate}
		\item Si $A$ est équicontinu en $x_{0}$, alors toute fonction $f \in A$
			est continue en $x_{0}$.
		\item Soit $B$ tel que $A \subseteq B$. Si $B$ est équicontinu en
			$x_{0}$ (resp. équicontinu), alors $A$ est équicontinu en $x_{0}$
			(resp. équicontinu).
		\item Si $A$ est équicontinu en $x_{0}$ (resp. équicontinu), alors
			$\adh{A}$ est équicontinu en $x_{0}$ (resp. équicontinu).
		\item Soit $(X^{2}, \tau_{X^{2}})$ où $\tau_{X^{2}}$ est la topologie
			produit. $A$ est équicontinu ssi pour tout $\epsilon > 0$, il existe un
			ouvert $O$ de $(X^{2}, \tau_{X^{2}})$ contenant la diagonale tel que
			pour tout $f \in A$, pour tout $(x, y) \in O$, $d_{Y}(f(x), f(y))
			\leq \epsilon$.
	\end{enumerate}
\end{proposition}

\ifdefined\outputproof
\begin{proof}

\end{proof}
\fi

\begin{proposition}
	Soient $(X, d_{X})$, $(Y, d_{Y})$ deux espaces métriques
	et soit $A \subseteq F(X, Y)$. Supposons que $(X, d_{X})$ est un espace
	métrique compact, alors les assertions suivantes sont équivalentes.

	\begin{enumerate}
		\item $A$ est équicontinu.
		\item $A$ est uniformément continu.
	\end{enumerate}
\end{proposition}

\ifdefined\outputproof
\begin{proof}

\end{proof}
\fi

\begin{proposition}
	Soient $(X, \tau)$ un espace topologique et soit $(Y, d_{Y})$ un espace métrique.
	Soit $(\mathcal{C}(X, Y), \tau_{\infty})$ où $\mathcal{C}(X, Y)$ est
	l'ensemble des fonctions continues de $(X, \tau)$ dans $(Y, d_{Y})$ et
	$\tau_{\infty}$ est la topologie de la convergence uniforme.

	Alors, la fonction
	\begin{equation}
		\phi : (\mathcal{C}(X, Y), \tau_{\infty}) \cartprod (X, \tau)
		\rightarrow (Y, d_{Y}) : (f, x) \rightarrow f(x)
	\end{equation}
	est continue.
\end{proposition}

\ifdefined\outputproof
\begin{proof}

\end{proof}
\fi

\begin{theorem} [Théorème d'Ascoli]
	Soient $(X, \tau)$ un espace topologique compact non vide, $(Y, d_{Y})$ un
	espace métrique. Soit $(\mathcal{C}(X, Y), \tau_{\infty})$ où
	$\mathcal{C}(X, Y)$ est l'ensemble des fonctions continues de $(X, \tau)$
	dans $(Y, d_{Y})$ et $\tau_{\infty}$ est la topologie de la convergence uniforme.

	Soit $A \subseteq \mathcal{C}(X, Y)$.

	Alors les assertions suivantes sont équivalentes.
	\begin{enumerate}
		\item $A$ est relativement compact dans $(\mathcal{C}(X, Y),
			\tau_{\infty})$
		\item $A$ est équicontinu et pour tout $x \in X$, $\GSsetDef{f(x)}{f \in
			A}$ est relativement compact dans $(Y, d_{Y})$
	\end{enumerate}
\end{theorem}

\ifdefined\outputproof
\begin{proof}

\end{proof}
\fi

\begin{corollary}
	Soient $(X, \tau)$ un espace topologique compact non vide, $(Y, d_{Y})$ un
	espace métrique. Soit $(\mathcal{C}(X, Y), \tau_{\infty})$ où
	$\mathcal{C}(X, Y)$ est l'ensemble des fonctions continues de $(X, \tau)$
	dans $(Y, d_{Y})$ et $\tau_{\infty}$ est la topologie de la convergence uniforme.

	Soit $A \subseteq \mathcal{C}(X, Y)$.

	Si $(Y, d_{Y})$ est compact, alors les assertions suivantes sont
	équivalentes.

	\begin{enumerate}
		\item $\adh{A}$ est compact dans $(\mathcal{C}(X, Y), \tau_{\infty})$.
		\item $A$ est équicontinu.
	\end{enumerate}
\end{corollary}

\ifdefined\outputproof
\begin{proof}

\end{proof}
\fi

Maintenant, oublions l'hypothèse de compacité sur $(Y, d_{Y})$, mais supposons
que $Y$ est un espace vectoriel normé. On obtient alors une autre équivalence
assez proche de la précédente.

\begin{corollary}
	Soient $(X, \tau)$ un espace topologique compact non vide, $(Y, d_{Y})$ un
	espace métrique. Soit $(\mathcal{C}(X, Y), \tau_{\infty})$ où
	$\mathcal{C}(X, Y)$ est l'ensemble des fonctions continues de $(X, \tau)$
	dans $(Y, d_{Y})$ et $\tau_{\infty}$ est la topologie de la convergence uniforme.

	Soit $A \subseteq \mathcal{C}(X, Y)$.

	Si $Y$ est un espace vectoriel de dimension finie, alors les assertions suivantes sont
	équivalentes.

	\begin{enumerate}
		\item $\adh{A}$ est compact dans $(\mathcal{C}(X, Y), \tau_{\infty})$.
		\item $A$ est équicontinu et borné.
	\end{enumerate}
\end{corollary}

\ifdefined\outputproof
\begin{proof}

\end{proof}
\fi
