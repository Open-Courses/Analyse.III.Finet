\chapter{Le théorème d'Ascoli}

\section{Rappels et définitions}

\begin{proposition}
	Soit $(X, \tau)$ un espace topologique.

	Soient $F$ un fermé de $(X, \tau)$ et $K$ un compact de $(X, \tau)$.

	Si $F \subseteq K$, alors $F$ est compact dans $(X, \tau)$.

	En d'autres termes, tout fermé dans un compact est compact.
\end{proposition}

\ifdefined\outputproof
\begin{proof}

\end{proof}
\fi

\begin{definition}
	Soit $(X, \tau)$ un espace topologique et soit $A \subseteq X$.

	On dit que $A$ est \textbf{relativement compact dans $(X, \tau)$} si
	$\adh{A}$ est compact dans $(X, \tau)$.
\end{definition}

\begin{proposition}
	Soit $(X, \tau)$ un espace topologique et soit $A \subseteq X$.
	Alors les assertions suivantes sont équivalentes.
	\begin{enumerate}
		\item $A$ est relativement compact dans $(X, \tau)$.
		\item $A$ est contenu dans un compact $K$ de $(X, \tau)$.
	\end{enumerate}
\end{proposition}

\ifdefined\outputproof
\begin{proof}

\end{proof}
\fi

\begin{definition}
	Soient $(X, \tau)$ un espace topologique et $(Y, d_{Y})$ un espace métrique.
	Posons
	\begin{equation}
		F(X, Y) = \GSset{f : X \rightarrow Y}
	\end{equation}
	l'ensemble des fonctions de $X$ dans $Y$.

	Soient $A \subseteq F(X, Y)$ et $x_{0} \in X$.
	On dit que $A$ est \textbf{équicontinu en $x_{0}$} si pour tout $\epsilon >
	0$, il existe un voisinage ouvert $V_{x_{0}}$ de $x_{0}$ tel que pour toute
	fonction $f \in A$, $d_{Y}(f(x), f(x_{0})) \leq \epsilon$.

	On dit que $A$ est \textbf{équicontinu} si $A$ est équicontinu en tout point
	$x_{0}$ de $X$.
\end{definition}

\begin{definition}
	Soient $(X, d_{X})$, $(Y, d_{Y})$ deux espaces métriques.
	Posons
	\begin{equation}
		F(X, Y) = \GSset{f : X \rightarrow Y}
	\end{equation}
	l'ensemble des fonctions de $X$ dans $Y$.

	Soit $A \subseteq F(X, Y)$.

	On dit que $A$ est \textbf{uniformément continu} si pour tout $\epsilon >
	0$, il existe $\delta > 0$ tel que pour tout $x \in X$, pour tout $y \in
	B(x, \delta)$ et pour tout $f \in A$, $d_{Y}(f(x), f(y)) \leq \epsilon$.
\end{definition}

On va munir $\mathcal{C}(X, Y)$, l'ensemble des fonctions continues de $(X,
\tau)$ dans $(Y, d_{Y})$, d'une topologie dépendante uniquement de
la métrique $d_{Y}$.

\begin{definition} [Topologie de la convergence uniforme]
	Soient $(X, \tau)$ un espace topologique compact et soit $(Y, d_{Y})$ un
	espace métrique.
	Soit $\mathcal{C}(X, Y)$ où $\mathcal{C}(X, Y)$ est
	l'ensemble des fonctions continues de $(X, \tau)$ dans $(Y, d_{Y})$

	\textbf{La topologie de la convergence uniforme}, notée $\tau_{\infty}$, est
	la topologie engendrée par la métrique
	\begin{equation}
		d : (\mathcal{C}(X, Y), \mathcal{C}(X, Y)) \rightarrow \real
	\end{equation}
	tel que
	\begin{equation}
		d(f, g) = \sup_{x \in X} d_{Y}(f(x), g(x))
	\end{equation}
\end{definition}

\begin{remarque}
	La définition de la métrique précédente contient un $\sup$. En realité,
	celui-ci est un maximum car $(X, \tau)$ est compact par hypothèse et $f, g$
	sont des fonctions continues.
\end{remarque}

\begin{exemple}
	\begin{enumerate}
		\item Soient $(X, d_{X})$ et $(Y, d_{Y})$ deux espaces métriques.
			L'ensemble des fonctions $k$-lipchitziennes pour $k > 0$ est
			équicontinu.
		\item Soient $(X, d_{X})$ et $(Y, d_{Y})$ deux espaces métriques. Soit
			\begin{equation}
				f : (X, d_{X}) \cartprod (X, d_{X}) \rightarrow (Y, d_{Y})
			\end{equation}
			une fonction uniformément continue et posons
			\begin{equation}
				A = \GSsetDef{f_{y} : X \rightarrow Y}{f_{y}(x) = f(x, y)}
			\end{equation}
			Alors $A$ est équicontinu.
	\end{enumerate}
\end{exemple}


\begin{proposition}
	Soient $(X, \tau)$ un espace topologique et $(Y, d_{Y})$ un espace métrique
	et soit $A \subseteq F(X, Y)$. Alors

	\begin{enumerate}
		\item Si $A$ est équicontinu en $x_{0}$, alors toute fonction $f \in A$
			est continue en $x_{0}$.
		\item Soit $B$ tel que $A \subseteq B$. Si $B$ est équicontinu en
			$x_{0}$ (resp. équicontinu), alors $A$ est équicontinu en $x_{0}$
			(resp. équicontinu).
		\item Si $A$ est équicontinu en $x_{0}$ (resp. équicontinu), alors
			$\adh{A}$ est équicontinu en $x_{0}$ (resp. équicontinu).
		\item Soit $(X^{2}, \tau_{X^{2}})$ où $\tau_{X^{2}}$ est la topologie
			produit. $A$ est équicontinu ssi pour tout $\epsilon > 0$, il existe un
			ouvert $O$ de $(X^{2}, \tau_{X^{2}})$ contenant la diagonale tel que
			pour tout $f \in A$, pour tout $(x, y) \in O$, $d_{Y}(f(x), f(y))
			\leq \epsilon$.
	\end{enumerate}
\end{proposition}

\ifdefined\outputproof
\begin{proof}

\end{proof}
\fi

\begin{proposition}
	Soient $(X, d_{X})$, $(Y, d_{Y})$ deux espaces métriques
	et soit $A \subseteq F(X, Y)$. Supposons que $(X, d_{X})$ est un espace
	métrique compact, alors les assertions suivantes sont équivalentes.

	\begin{enumerate}
		\item $A$ est équicontinu.
		\item $A$ est uniformément continu.
	\end{enumerate}
\end{proposition}

\ifdefined\outputproof
\begin{proof}

\end{proof}
\fi

\begin{proposition}
	Soient $(X, \tau)$ un espace topologique et soit $(Y, d_{Y})$ un espace métrique.
	Soit $(\mathcal{C}(X, Y), \tau_{\infty})$ où $\mathcal{C}(X, Y)$ est
	l'ensemble des fonctions continues de $(X, \tau)$ dans $(Y, d_{Y})$ et
	$\tau_{\infty}$ est la topologie de la convergence uniforme.

	Alors, la fonction d'évalutation
	\begin{equation}
		\phi : (\mathcal{C}(X, Y), \tau_{\infty}) \cartprod (X, \tau)
		\rightarrow (Y, d_{Y}) : (f, x) \rightarrow f(x)
	\end{equation}
	est continue.
\end{proposition}

\ifdefined\outputproof
\begin{proof}
	Il faut montrer que pour tout $(f_{0}, x_{0})$, pour tout $\epsilon > 0$,
	il existe $V_{(f_{0}, x_{0})}$ tel que pour tout $(f, x) \in V_{(f_{0},
	x_{0})}$, $d_{Y}(f_{0}(x_{0}), f(x)) \leq \epsilon$.

	Soit $(f_{0}, x_{0})$ et $\epsilon > 0$. On sait que $f_{0}$ est continue en
	$x_{0}$ par hypothèse. Donc
	\begin{equation}
		\forall \epsilon' > 0, \, \exists V_{x_{0}}, \, \forall x \in V_{x_{0}},
		\, d_{Y}(f_{0}(x), f_{0}(x_{0})) \leq \epsilon'
	\end{equation}
	Prenons le cas particulier où $\epsilon' = \frac{\epsilon}{2}$ et prenons
	\begin{equation}
		V_{(f_{0}, x_{0})} = B_{\mathcal{C}(X, Y)}(f_{0}, \frac{\epsilon}{2})
		\cartprod V_{x_{0}}
	\end{equation}
	où $V_{x_{0}}$ est le voisinage de $x_{0}$ donné par $\epsilon'$.

	Il nous reste à vérifier que pour tout $(f, x) \in V_{(f_{0}, x_{0})}$,
	$d_{Y}(f(x), f_{0}(x_{0})) \leq \epsilon$.

	Soit $(f, x) \in V_{(f_{0}, x_{0})}$. On a
	\begin{align}
		d_{Y}(f(x), f_{0}(x_{0})) & \leq d(f(x), f(x_{0})) + d(f(x_{0}), f_{0}(x_{0}))
		\\
		& \leq \frac{\epsilon}{2} + \frac{\epsilon}{2} \\
		& \leq \epsilon
	\end{align}
\end{proof}
\fi

\section{Enoncé du théorème et corollaires}

\begin{theorem} [Théorème d'Ascoli]
	Soient $(X, \tau)$ un espace topologique compact non vide, $(Y, d_{Y})$ un
	espace métrique. Soit $(\mathcal{C}(X, Y), \tau_{\infty})$ où
	$\mathcal{C}(X, Y)$ est l'ensemble des fonctions continues de $(X, \tau)$
	dans $(Y, d_{Y})$ et $\tau_{\infty}$ est la topologie de la convergence uniforme.

	Soit $A \subseteq \mathcal{C}(X, Y)$.

	Alors les assertions suivantes sont équivalentes.
	\begin{enumerate}
		\item $A$ est relativement compact dans $(\mathcal{C}(X, Y),
			\tau_{\infty})$
		\item $A$ est équicontinu et pour tout $x \in X$, $\GSsetDef{f(x)}{f \in
			A}$ est relativement compact dans $(Y, d_{Y})$
	\end{enumerate}
\end{theorem}

\ifdefined\outputproof
\begin{proof}

\end{proof}
\fi

On remarque qu'en ajoutant l'hypothèse de compacité sur $(Y, d_{Y})$, on obtient
immédiatement que $\adh{\GSsetDef{f(x)}{f \in A}}$ est relativement compact dans
$(Y, d_{Y})$.

\begin{corollary}
	Soient $(X, \tau)$ un espace topologique compact non vide, $(Y, d_{Y})$ un
	espace métrique. Soit $(\mathcal{C}(X, Y), \tau_{\infty})$ où
	$\mathcal{C}(X, Y)$ est l'ensemble des fonctions continues de $(X, \tau)$
	dans $(Y, d_{Y})$ et $\tau_{\infty}$ est la topologie de la convergence uniforme.

	Soit $A \subseteq \mathcal{C}(X, Y)$.

	Si $(Y, d_{Y})$ est compact, alors les assertions suivantes sont
	équivalentes.

	\begin{enumerate}
		\item \label{statement:adherence} $\adh{A}$ est compact dans $(\mathcal{C}(X, Y), \tau_{\infty})$.
		\item \label{statement:equicontinu} $A$ est équicontinu.
	\end{enumerate}
\end{corollary}

\ifdefined\outputproof
\begin{proof}
	$(\ref{statement:adherence}) \implies (\ref{statement:equicontinu})$.
	On a vu que $\adh{A}$ compact dans $(\mathcal{C}(X, Y), \tau_{\infty})$
	est équivalent à $A$ relativement compact dans $(\mathcal{C}(X, Y),
	\tau_{\infty})$.

	Par le théorème d'Ascoli, $A$ est équicontinu.

	$(\ref{statement:equicontinu}) \implies (\ref{statement:adherence})$.
	Il suffit de montrer, par le théorème d'Ascoli, que pour tout $x \in X$
	\begin{equation}
		\GSsetDef{f(x)}{f \in A}
	\end{equation}
	est relativement compact dans $(Y, d_{Y})$,
	c'est-à-dire que
	\begin{equation}
		\adh{\GSsetDef{f(x)}{f \in A}}
	\end{equation}
	est compact dans $(Y, d_{Y})$.

	Comme $\adh{\GSsetDef{f(x)}{f \in A}}$ est fermé dans $(Y, d_{Y})$ et que
	$Y$ est compact dans $(Y, d_{Y})$, alors $\adh{\GSsetDef{f(x)}{f \in A}}$ est
	compact dans $(Y, d_{Y})$.
\end{proof}
\fi

Le théorème d'Ascoli donne donc une équivalence entre la compactité et
l'équicontinuité quand nous avons deux espaces compacts.

Maintenant, oublions l'hypothèse de compacité sur $(Y, d_{Y})$, mais supposons
que $Y$ est un espace vectoriel normé. On obtient alors une autre équivalence
assez proche de la précédente.

\begin{corollary}
	Soient $(X, \tau)$ un espace topologique compact non vide, $(Y, d_{Y})$ un
	espace métrique. Soit $(\mathcal{C}(X, Y), \tau_{\infty})$ où
	$\mathcal{C}(X, Y)$ est l'ensemble des fonctions continues de $(X, \tau)$
	dans $(Y, d_{Y})$ et $\tau_{\infty}$ est la topologie de la convergence uniforme.

	Soit $A \subseteq \mathcal{C}(X, Y)$.

	Si $Y$ est un espace vectoriel de dimension finie, alors les assertions suivantes sont
	équivalentes.

	\begin{enumerate}
		\item $\adh{A}$ est compact dans $(\mathcal{C}(X, Y), \tau_{\infty})$.
		\item $A$ est équicontinu et borné.
	\end{enumerate}
\end{corollary}

\ifdefined\outputproof
\begin{proof}

\end{proof}
\fi
