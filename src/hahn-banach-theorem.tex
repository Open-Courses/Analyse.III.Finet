\chapter{Théorèmes de Hahn-Banach et applications}

Les théorèmes de Hahn-Banach s'intéressent aux formes linéaires sur les espaces
vectoriels normés sur les corps $\real$ ou $\complex$ à travers leurs formes
linéaires, et aux hyperplans avec leurs formes géométriques.

Ceux-ci permettent de montrer l'existence de prolongement des formes linéaires
définies au départ sur un sous-espace vectoriel.

Nous les montrerons d'abord sur $\real$, et après sur $\complex$ (qui sera une
généralisation simple).

\section{Formes analytiques}

\begin{theorem} [Théorème 1 de Hahn-Banach --- analytique]
\label{theorem_hahn_banach_analytic_1}
	Soit E un $\real$-espace vectoriel, et G un sous-espace vectoriel de E.

	Soit $\GSfunction{f_{G}}{G}{\real}$ une forme linéaire, et
	$\GSfunction{p}{E}{\real}$ une fonction convexe.

	Supposons que
	\begin{equation}
		\forall x \in G, \abs{f_{G}(x)} \leq p(x)
	\end{equation}
	c'est-à-dire que $f_{G}$ est majorée par $p$ sur $G$.

	Alors il existe une forme linéaire
	\begin{equation}
		\GSfunction{f_{E}}{E}{\real}
	\end{equation}
	tel que
	\begin{enumerate}
		\item ${(f_{E})}_{|G} = f_{G}$
		\item $\forall x \in E$, $\abs{f_{E}(x)} \leq p(x)$.
	\end{enumerate}
\end{theorem}

La dernière condition insiste bien sur le fait que $f_{E}$ prolonge $f_{G}$ et
que $f_{E}$ reste majorée par $p$, mais cette fois-ci, sur tout $E$, et pas
seulement sur le sous espace vectoriel $G$.
Remarquons en plus que $E$ n'est pas nécessairement normé. En effet, nous aurons
un corollaire portant sur les espaces vectoriels normés.

De plus, $f$ n'est pas nécessairement continue.

\ifdefined\outputproof
\begin{proof}

\end{proof}
\fi

\begin{corollary}
	Si de plus $E$ est \textbf{normé}, on a $||f_{E}|| = ||f_{G}||$.
\end{corollary}

\ifdefined\outputproof
\begin{proof}

\end{proof}
\fi

Nous pouvons nous demander si la condition de majoration par une fonction
convexe est nécessaire. En réalité, si on suppose que $f$ est linéaire et
continue, nous pouvons contruire $p(x)$, et nous avons alors le théorème
suivant.
%Et majoration nécessaire si pas continue ?

\begin{theorem} [Théorème 2 de Hahn Banach --- analytique]
\label{theorem_hahn_banach_analytic_2}
	Soit E un $\real$-espace vectoriel \textbf{normé}, et G un sous-espace
	vectoriel de E.
	Soit $\GSfunction{f_{G}}{G}{\real}$ une forme linéaire
	\textbf{continue}.

	Alors il existe une forme linéaire \textbf{continue}
	$\GSfunction{f_{E}}{E}{\real}$ tel que $(f_{E})_{|G} = f_{G}$.
\end{theorem}

\ifdefined\outputproof
\begin{proof}
	On va construire $p$ qui majore $f$.
	On pose
	\begin{equation}
		\GSfunction{p}{E}{\real} : x \rightarrow \GSnorme{f} \,
		\GSnormeDef{x}{E}
	\end{equation}

	Comme $f$ est linéaire continue sur $G$, on a bien
	\begin{equation}
		|f(x)| \leq ||f|| \, ||x||_{E} = p(x)
	\end{equation}

	Donc on peut étendre $f$ en une forme linéaire $f_{E}$
	sur tout $E$ par~\ref{theorem_hahn_banach_analytic_1}, et tel que
	\begin{equation}
		\abs{f_{E}(x)} \leq p(x) = \GSnorme{f_{G}} \, \GSnormeDef{x}{E} =
			\GSnorme{f_{E}} \, \GSnormeDef{x}{E}
	\end{equation}
	c'est-à-dire $f_{E}$ continue.
\end{proof}
\fi
\section{Applications des formes analytiques}

Dans cette section, nous allons démontrer quelques corollaires des théorèmes de
Hahn-Banach analytiques.

\begin{corollary}
	Soient $\GSnormedSpace{E}{\GSnormeDef{.}{E}}$ un espace vectoriel normé et $x_{0}$
	un élément non-nul de $E$.

	Alors, il existe $x^{*} \in \GSdual{E}$ tel que
	\begin{enumerate}
		\item $x^{*}(x_{0}) = 1$
		\item $\GSnormeDef{x^{*}}{\GSdual{E}} = \frac{1}{\GSnormeDef{x_{0}}{E}}$
	\end{enumerate}
\end{corollary}

\ifdefined\outputproof
\begin{proof}
	Posons $G = \spanspace{x_{0}}$, et
	\begin{equation}
		\GSfunction{\tilde{x}^{*}}{G}{\real} : \lambda x_{0} \rightarrow \lambda
	\end{equation}
	On a $\tilde{x}^{*}$ qui est une forme linéaire continue. Il existe donc un
	prolongement $x^{*} \in \GSdual{E}$ de $\tilde{x}^{*}$.

	De plus, on a bien que $x^{*}(x_{0}) = 1$.

	Enfin,
	\begin{align}
		\GSnormeDef{x^{*}}{\GSdual{E}} &= \GSnormeDef{\tilde{x}^{*}}{\GSdual{G}} \\
		&= \sup_{\GSnormeDef{\lambda x_{0}}{E} \leq 1} \abs{\lambda} \\
		&= \sup_{\abs{\lambda} \GSnormeDef{x_{0}}{E} \leq 1} \abs{\lambda} \\
		&= \sup_{|\lambda| \leq \frac{1}{\GSnormeDef{x_{0}}{E}}} \abs{\lambda} \\
		&= \frac{1}{\GSnormeDef{x_{0}}{E}}
	\end{align}
\end{proof}
\fi

\begin{corollary}
	Soient $\GSnormedSpace{E}{\GSnormeDef{.}{E}}$ et $x_{0}$ un élément non-nul de $E$.

	Alors, il existe $x^{*} \in \GSdual{E}$ tel que
	\begin{enumerate}
		\item $x^{*}(x_{0}) = \GSnormeDef{x_{0}}{E}$
		\item $\GSnorme{x^{*}} = \GSnormeDef{x_{0}}{E}$
	\end{enumerate}
\end{corollary}

\ifdefined\outputproof
\begin{proof}
	Même que le corollaire précédent où $x^{*}$ est la norme restreinte à
	$\spanspace{x_{0}}$.
\end{proof}
\fi

\begin{corollary}
	Soit $\GSnormedSpace{E}{\GSnormeDef{.}{E}}$. Alors $\GSdual{E}$ sépare les points de
	E, c'est-à-dire que pour tout $x, y \in E$, il existe $x^{*} \in \GSdual{E}$
	tel que $x^{*}(x) \neq x^{*}(y)$.
\end{corollary}

\ifdefined\outputproof
\begin{proof}

\end{proof}
\fi

\begin{corollary}
	Soit $\GSnormedSpace{E}{\GSnormeDef{.}{E}}$. Alors pour tout $x \in E$,
	\begin{equation}
		\GSnormeDef{x}{E} = \max\limits_{x^{*} \in \GSunitBoule{\GSdual{E}}} \abs{x^{*}(x)}
	\end{equation}
\end{corollary}

\ifdefined\outputproof
\begin{proof}

\end{proof}
\fi

\begin{corollary}
	Soit $\GSnormedSpace{E}{\GSnormeDef{.}{E}}$, et $H$ un sous-espace vectoriel fermé.
	Soit $x_{0} \in E \backslash H$.

	Alors il existe $x^{*} \in \GSdual{E}$ tel que :

	\begin{enumerate}
		\item $\GSnorme{x^{*}} = 1$.
		\item $x^{*}(x_{0}) = d(x_{0}, H)$
		\item $H \subseteq ker(x^{*})$
	\end{enumerate}
\end{corollary}

\ifdefined\outputproof
\begin{proof}

\end{proof}
\fi

\section{Formes géométriques}

Donnons d'abord quelques définitions. Dans cette partie, $E$ désigne un espace
vectoriel.

\begin{definition} [Jauge]
	Soit $C \subseteq E$ tel que $0 \in C$.

	On définit \textbf{la jauge de $C$} comme la fonction
	\begin{equation}
		\GSfunction{p}{E}{\real^{+}} : x \rightarrow \inf \GSsetDef{\alpha >
		0}{\alpha^{-1}x \in C}
	\end{equation}
\end{definition}

\begin{exercice}
	$p(0) = 0$ quelque soit $C$.
\end{exercice}
%Donner des exercices et des exemples, avec les corrigés

Regardons quels résultats nous avons quand $C$ est lié à la topologie de $E$.

\begin{proposition}
	Si $C$ est ouvert, alors il existe $M > 0$ tel que pour tout $x \in E$
	\begin{equation}
		p(x) \leq M \GSnormeDef{x}{E}
	\end{equation}
\end{proposition}

\ifdefined\outputproof
\begin{proof}

	Comme $C$ est ouvert et que $0 \in C$ par hypothèse, il existe $r > 0$ tel
	que $B(0, r) \subseteq C$.

	Prenons $x \in E$ non nul (le cas $x = 0$ étant trivial).
	Alors, en posant
	\begin{equation}
		\alpha^{-1} = \frac{r}{2} \frac{1}{||x||}
	\end{equation}
	on a $\alpha^{-1} x \in B(0, r)$.
	Donc $\alpha^{-1} x \in C$.

	Par conséquent,
	\begin{equation}
		p(x) \leq \alpha = \frac{2}{r} ||x||
	\end{equation}
\end{proof}
\fi

\begin{proposition}
	Si $C$ est ouvert et convexe, alors
	\begin{equation}
		C = \GSsetDef{x \in E}{p(x) < 1}
	\end{equation}
\end{proposition}

\ifdefined\outputproof
\begin{proof}

\end{proof}
\fi

\begin{definition} [Hyperplan]
	Soit $H$ un sous-espace vectoriel de $E$.

	On dit que $H$ est \textbf{un hyperplan (vectoriel)} s'il est de codimension
	1, c'est-à-dire que $E = H \oplus \mathbb{K}e$ où $e \in E$ et $e$ non nul.
\end{definition}

\begin{proposition}
	Les assertions suivantes sont équivalentes.
	\begin{enumerate}
		\item \label{statement:h_hyperplan} $H$ est un hyperplan.
		\item \label{statement:kernel_linear_form} il existe une forme linéaire $\GSfunction{f}{E}{\mathbb{K}}$ non nulle tel que $H = \ker(f)$.
	\end{enumerate}
	En d'autres termes, tout hyperplan est le noyau d'une forme linéaire non
	nulle.
\end{proposition}

\ifdefined\outputproof
\begin{proof}
	$(\ref{statement:h_hyperplan} \implies \ref{statement:kernel_linear_form})$
	Par hypothèse, on sait qu'il existe $e \in E$ non nul tel que $E = H \oplus \mathbb{K}e$.

	Posons
	\begin{equation}
		\GSfunction{f}{\overbrace{H \oplus \mathbb{K}e}^{E}}{\mathbb{K}} : h + \lambda e \rightarrow \lambda
	\end{equation}
	$f$ est une forme linéaire continue et $H = \ker(f)$.

	$(\ref{statement:kernel_linear_form} \implies \ref{statement:h_hyperplan})$
	Soit $x \in E$, et $e \notin \ker(f)$.

	On peut alors écrire $x$ tel que $x = (x - \lambda e) + \lambda e$.

	Il nous faut $\lambda$ tel que $x - \lambda e \in
	\ker(f)$.

	On a $f(x - \lambda e) = f(x) - \lambda f(e)$.
	Si
	\begin{equation}
		\lambda = \frac{f(x)}{f(e)}
	\end{equation}
	on a $x - \lambda e \in \ker(f)$.
	
	On a donc bien que $\ker(f)$ est un hyperplan vectoriel car $E = \ker(f)
	\oplus \mathbb{K}e$.
\end{proof}
\fi

\begin{definition} [Hyperplan affine]
	Soit $\GSfunction{f}{E}{\mathbb{K}_{0}}$, et $\alpha \in \mathbb{K}$.

	Soit $H = \GSsetDef{x \in E}{f(x) = \alpha}$.

	$H$ est appelé \textbf{hyperplan (vectoriel) affine}.
\end{definition}


\begin{definition}
	Soient $A$, $B \subseteq E$ deux ensembles convexes et disjoints.

	Soit $H = \GSsetDef{x \in E}{f(x) = \alpha}$ un hyperplan affine.

	On dit que \textbf{$H$ sépare $A$ et $B$ au sens large} si

	\begin{enumerate}
		\item pour tout $x \in A$, $f(x) \leq \alpha$
		\item pour tout $x \in B$, $f(x) \geq \alpha$
	\end{enumerate}

	et on dit que \textbf{$H$ sépare $A$ et $B$ au sens strict} s'il existe
	$\epsilon > 0$ tel que

	\begin{enumerate}
		\item pour tout $x \in A$, $f(x) \leq \alpha -
	\epsilon$
		\item pour tout $x \in B$, $f(x) \geq \alpha + \epsilon$.
	\end{enumerate}
\end{definition}

Nous en venons aux théorèmes de Hahn-Banach sous leurs formes géométriques.

\begin{theorem} [Théorème de Hahn-Banach 1 --- géométrique]
	Soient $A$, $B \subseteq E$ deux ensembles convexes et disjoints tel que $A$ est ouvert.

	Alors il existe un hyperplan vectoriel fermé qui sépare $A$ et $B$ au sens
	large.
\end{theorem}

\ifdefined\outputproof
\begin{proof}

\end{proof}
\fi

\begin{theorem} [Théorème de Hahn-Banach 2 --- géométrique]
	Soit $A$, $B \subseteq E$ convexes et disjoints tel que $A$ est fermé et $B$
	compact.

	Alors il existe un hyperplan vectoriel fermé qui sépare $A$ et $B$ au sens
	strict.
\end{theorem}

\ifdefined\outputproof
\begin{proof}

\end{proof}
\fi

\section{Applications géométriques}

\begin{corollary}
	Soient deux convexes $A$ et $B$ tel que $A-B$ est dense dans E.

	Alors $A$ et $B$ ne peuvent être séparés par un hyperplan fermé.
\end{corollary}

\ifdefined\outputproof
\begin{proof}
	Supposons qu'il existe un hyperplan fermé qui les sépare, c'est-à-dire qu'il
	existe $x^{*} \in E^{*}$ non nulle et $\alpha \in \mathbb{K}$ tel que
	\begin{equation}
		\forall a \in A, \, x^{*}(a) \leq \alpha
	\end{equation}
	et
	\begin{equation}
		\forall b \in B, \, x^{*}(b) \geq \alpha
	\end{equation}

	Soit $x \in A - B$ tel que $x^{*}(x) \neq 0$. Sans perte de généralité,
	$x^{*}(x) > 0$
	\footnote{Si $x^{*}(x) < 0$, faire le même raisonnement avec $-x$. On aura
	bien $x^{*}(-x) = -x^{*}(x) > 0$.}.
	
	Comme $A - B$ est dense dans $E$, il existe une suite $(a_{n} - b_{n}) \rightarrow x$.

	On a pour tout $n \in \naturel$, $x^{*}(a_{n}) \leq \alpha$ et $x^{*}(b_{n})
	\geq \alpha$, donc $x^{*}(a_{n} - b_{n}) = x^{*}(a_{n}) - x^{*}(b_{n}) \leq
	0$.

	Or, comme $x^{*}$ est continue, $x^{*}(a_{n} - b_{n}) \rightarrow x^{*}(x)
	\leq 0$. Contradiction avec l'hypothèse que $x^{*}(x) > 0$.
\end{proof}
\fi

\begin{corollary}
	Tout hyperplan vectoriel est soit fermé, soit dense dans $E$.
\end{corollary}

\ifdefined\outputproof
\begin{proof}

\end{proof}
\fi
