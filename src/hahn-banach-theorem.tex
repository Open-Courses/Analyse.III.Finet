\section{Théorèmes de Hahn-Banach et applications}

Les théorèmes de Hahn-Banach s'intéresse aux formes linéaires sur les espaces
vectoriels normés sur les corps $\real$ ou $\complex$.
Ceux-ci permettent de montrer l'existence de prolongement des formes linéaires
définies au départ sur un sous-espace vectoriel.

Nous les montrerons d'abord sur $\real$, et après sur $\complex$ (qui sera une
généralisons simple).

\subsection{Forme analytique}

\begin{theorem} [Théorème 1 de Hahn-Banach - analytique]
	Soit E un $\real$-espace vectoriel, et G un sous-espace vectoriel de E.

	Soit \GSfunction{$f_{G}$}{$G$}{$\real$} une forme linéaire, et
	\GSfunction{$p$}{$E$}{$\real$} une fonction convexe.

	Supposons que $\forall x \in G$ $f_{G}(x) \leq p(x)$, c'est-à-dire que
	$f_{G}$ est majorée par $p$ sur $G$.

	Alors il existe une forme linéaire \GSfunction{$f_{E}$}{$E$}{$\real$} tel que
	$(f_{E})_{|G} = f_{G}$ et $\forall x \in E$ $f_{E}(x) \leq p(x)$.
	\label{hahn-banach-analytic-1}
\end{theorem}

La dernière condition insiste bien sur le fait que $f_{E}$ prolonge $f_{G}$ et
que $f_{E}$ reste majorée par $p$, mais cette fois-ci, sur tout $E$, et pas
seulement sur le sous espace vectoriel $G$.
Remarquons en plus que $E$ n'est pas nécessairement normé. En effet, nous aurons
un corollaire portant sur les espaces vectoriels normés.

De plus, $f$ n'est pas nécessairement continue.

\begin{proof}
	
\end{proof}

\begin{corollary}
	Si de plus $E$ est \textbf{normé}, on a $||f_{E}|| = ||f_{G}||$.
\end{corollary}

Nous pouvons nous demander si la condition de majoration par une fonction
convexe est nécessaire. En réalité, si on suppose que $f$ est linéaire et
continue, nous pouvons contruire $p(x)$, et nous avons alors le théorème
suivant.

\begin{theorem} [Théorème 2 de Hahn Banach - analytique]
	Soit E un $\real$-espace vectoriel \textbf{normé}, et G un sous-espace
	vectoriel de E.
	Soit \GSfunction{$f_{G}$}{$G$}{$\real$} une forme linéaire
	\textbf{continue}.

	Alors il existe une forme linéaire \textbf{continue}
	\GSfunction{$f_{E}$}{$E$}{$\real$} tel que $(f_{E})_{|G} = f_{G}$.
	\label{hahn-banach-analytic-2}
\end{theorem}

\subsection{Applications analytiques}

Dans cette section, nous allons démontrer quelques corollaires des théorèmes de
Hahn-Banach analytiques.

\begin{corollary}
	Soit \GSnormedSpace{E}{\GSnorme{.}}, et un élément non-nul $x_{0}$ de $E$.

	Alors, $\exists x^{*} \in \GSdual{E}$ tel que $x^{*}(x_{0}) = 1$ et
	\GSnorme{x^{*}} $= \frac{1}{||x_{0}||}$
\end{corollary}

\begin{proof}
	
\end{proof}

\begin{corollary}
	Soit \GSnormedSpace{E}{\GSnorme{.}}, et un élément non-nul $x_{0}$ de $E$.

	Alors, $\exists x^{*} \in \GSdual{E}$ tel que $x^{*}(x_{0}) = x_{0}$ et
	\GSnorme{x^{*}} $= ||x_{0}||$.
\end{corollary}

\begin{proof}
	
\end{proof}

\begin{corollary}
	Soit \GSnormedSpace{E}{\GSnorme{.}}. Alors $\GSdual{E}$ sépare les points de
	E, c'est-à-dire que $\forall x, y \in E$ $\exists x^{*} \in \GSdual{E}$ tel
	que $x^{*}(x) \neq x^{*}(y)$.
\end{corollary}

\begin{proof}
	
\end{proof}

\begin{corollary}
	Soit \GSnormedSpace{E}{\GSnorme{.}}. Alors $\forall x \in E$, \GSnorme{x} 
	$ = \max\limits_{x^{*} \in \GSunitBoule{\GSdual{E}}}$\GSnorme{x^{*}(x)}.
\end{corollary}

\begin{proof}
	
\end{proof}

\begin{corollary}
	Soit \GSnormedSpace{E}{\GSnorme{.}}, et $H$ un sous-espace vectoriel fermé.
	Soit $x_{0} \in E_{/H}$.

	Alors $\exists x^{*} \in \GSdual{E}$ tel que :

	\begin{enumerate}
		\item \GSnorme{x^{*}} $ = 1$.
		\item $x^{*}(x_{0}) = d(x_{0}, H)$
		\item $H \subseteq ker(x^{*})$
	\end{enumerate}
\end{corollary}

\subsection{Forme géométrique}

