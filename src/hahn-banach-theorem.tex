\section{Théorèmes de Hahn-Banach et applications}

Les théorèmes de Hahn-Banach s'intéresse aux formes linéaires sur les espaces
vectoriels normés sur les corps $\real$ ou $\complex$.
Ceux-ci permettent de montrer l'existence de prolongement des formes linéaires
définies au définies au départ sur un sous-espace vectoriel.

Nous les montrerons d'abord sur $\real$, et après sur $\complex$ (qui sera une
généralisons simple).

Un des théorèmes, dit analytique est le suivant :

\begin{theorem} [Hahn-Banach - analytique]
	Soit E un $\real$-espace vectoriel normé, et G un sous-espace vectoriel de E.
	Soit \GSfunction{$f_{G}$}{G}{$\real$} une forme linéaire.

	Alors il existe une forme linéaire \GSfunction{$f_{E}$}{E}{$\real$} tel que
	$(f_{E})_{|G} = f_{G}$.
\end{theorem}

La dernière condition insiste bien sur le fait que $f_{E}$ prolonge $f_{G}$

\begin{proof}
	
\end{proof}


