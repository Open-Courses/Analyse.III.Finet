\chapter{Théorèmes de Hahn-Banach et applications}

Les théorèmes de Hahn-Banach s'intéressent aux formes linéaires sur les espaces
vectoriels normés sur les corps $\real$ ou $\complex$ à travers leurs formes
linéaires, et aux hyperplans avec leurs formes géométriques.

Ceux-ci permettent de montrer l'existence de prolongement des formes linéaires
définies au départ sur un sous-espace vectoriel.

Nous les montrerons d'abord sur $\real$, et après sur $\complex$ (qui sera une
généralisation simple).

\section{Forme analytique}

\begin{theorem} [Théorème 1 de Hahn-Banach --- analytique]
\label{theorem_hahn_banach_analytic_1}
	Soit E un $\real$-espace vectoriel, et G un sous-espace vectoriel de E.

	Soit $\GSfunction{f_{G}}{G}{\real}$ une forme linéaire, et
	$\GSfunction{p}{E}{\real}$ une fonction convexe.

	Supposons que $\forall x \in G$ $f_{G}(x) \leq p(x)$, c'est-à-dire que
	$f_{G}$ est majorée par $p$ sur $G$.

	Alors il existe une forme linéaire $\GSfunction{f_{E}}{E}{\real}$ tel que
	${(f_{E})}_{|G} = f_{G}$ et $\forall x \in E$ $f_{E}(x) \leq p(x)$.
\end{theorem}

La dernière condition insiste bien sur le fait que $f_{E}$ prolonge $f_{G}$ et
que $f_{E}$ reste majorée par $p$, mais cette fois-ci, sur tout $E$, et pas
seulement sur le sous espace vectoriel $G$.
Remarquons en plus que $E$ n'est pas nécessairement normé. En effet, nous aurons
un corollaire portant sur les espaces vectoriels normés.

De plus, $f$ n'est pas nécessairement continue.

\ifdefined\outputproof
\begin{proof}

\end{proof}
\fi

\begin{corollary}
	Si de plus $E$ est \textbf{normé}, on a $||f_{E}|| = ||f_{G}||$.
\end{corollary}

\ifdefined\outputproof
\begin{proof}

\end{proof}
\fi

Nous pouvons nous demander si la condition de majoration par une fonction
convexe est nécessaire. En réalité, si on suppose que $f$ est linéaire et
continue, nous pouvons contruire $p(x)$, et nous avons alors le théorème
suivant.
%Et majoration nécessaire si pas continue ?

\begin{theorem} [Théorème 2 de Hahn Banach --- analytique]
\label{theorem_hahn_banach_analytic_2}
	Soit E un $\real$-espace vectoriel \textbf{normé}, et G un sous-espace
	vectoriel de E.
	Soit $\GSfunction{f_{G}}{G}{\real}$ une forme linéaire
	\textbf{continue}.

	Alors il existe une forme linéaire \textbf{continue}
	$\GSfunction{f_{E}}{E}{\real}$ tel que $(f_{E})_{|G} = f_{G}$.
\end{theorem}

\ifdefined\outputproof
\begin{proof}
	On va construire $p$ qui majore $f$.
	On pose $\GSfunction{p}{E}{\mathbb{K}}$ : $x \rightarrow ||f|| \,
	||x||_{E}$.

	Comme $f$ est linéaire continue sur $G$, on a bien que $|f(x)| \leq ||f|| \,
	||x||_{E} = p(x)$. Donc on peut étendre $f$ en une forme linéaire $f_{E}$
	sur tout $E$ par~\ref{theorem_hahn_banach_analytic_1}, et tel que
	$|f_{E}(x)| \leq p(x) = ||f_{G}|| \, ||x||_{E} = ||f_{E}|| \, ||x||_{E}$,
	c'est-à-dire $f_{E}$ continue.
\end{proof}
\fi
\section{Applications analytiques}

Dans cette section, nous allons démontrer quelques corollaires des théorèmes de
Hahn-Banach analytiques.

\begin{corollary}
	Soit $\GSnormedSpace{E}{\GSnorme{.}}$, et un élément non-nul $x_{0}$ de $E$.

	Alors, $\exists x^{*} \in \GSdual{E}$ tel que $x^{*}(x_{0}) = 1$ et
	$||x^{*}||_{E^{*}} = \frac{1}{||x_{0}||}$
\end{corollary}

\ifdefined\outputproof
\begin{proof}
	Prenons $G = <x_{0}>$, et $\GSfunction{x^{*}}{G}{\mathbb{K}}$ : $\lambda
	x_{0} \rightarrow \lambda$. On a bien $x^{*}$ linéaire et continue, donc on peut
	la prolonger sur tout $E$, et donc $x^{*} \in \GSdual{E}$.

	De plus, on a bien que $x^{*}(x_{0}) = 1$.

	Enfin, $||x^{*}||_{E^{*}} = ||x^{*}||_{G} = \displaystyle \sup_{||\lambda x_{0}|| \leq
1}|\lambda| = \sup_{|\lambda|||x_{0}|| \leq 1} = \sup_{|\lambda| \leq
	\frac{1}{||x_{0}||}} = \frac{1}{||x_{0}||}$.
\end{proof}
\fi

\begin{corollary}
	Soit $\GSnormedSpace{E}{\GSnorme{.}}$, et un élément non-nul $x_{0}$ de $E$.

	Alors, $\exists x^{*} \in \GSdual{E}$ tel que $x^{*}(x_{0}) = ||x_{0}||$ et
	$\GSnorme{x^{*}} = ||x_{0}||$.
\end{corollary}

\ifdefined\outputproof
\begin{proof}
	Même que le corollaire précédent où $x_{*}$ est la norme restreinte à
	$<x_{0}>$.
\end{proof}
\fi

\begin{corollary}
	Soit $\GSnormedSpace{E}{\GSnorme{.}}$. Alors $\GSdual{E}$ sépare les points de
	E, c'est-à-dire que $\forall x, y \in E$ $\exists x^{*} \in \GSdual{E}$ tel
	que $x^{*}(x) \neq x^{*}(y)$.
\end{corollary}

\ifdefined\outputproof
\begin{proof}

\end{proof}
\fi

\begin{corollary}
	Soit $\GSnormedSpace{E}{\GSnorme{.}}$. Alors $\forall x \in E$, $\GSnorme{x}
	= \max\limits_{x^{*} \in \GSunitBoule{\GSdual{E}}} \GSnorme{x^{*}(x)}$.
\end{corollary}

\ifdefined\outputproof
\begin{proof}

\end{proof}
\fi

\begin{corollary}
	Soit $\GSnormedSpace{E}{\GSnorme{.}}$, et $H$ un sous-espace vectoriel fermé.
	Soit $x_{0} \in E_{/H}$.

	Alors $\exists x^{*} \in \GSdual{E}$ tel que :

	\begin{enumerate}
		\item $\GSnorme{x^{*}} = 1$.
		\item $x^{*}(x_{0}) = d(x_{0}, H)$
		\item $H \subseteq ker(x^{*})$
	\end{enumerate}
\end{corollary}

\section{Forme géométrique}

Donnons d'abord quelques définitions. Dans cette partie, $E$ désigne un espace
vectoriel.

\begin{definition} [Jauge]
	Soit $C \subseteq E$ tel que $0 \in C$.

	La jauge de $C$ est définie comme la fonction $\GSfunction{p}{E}{R^{+}}$
	tel que $p(x) = \inf\left\{ \alpha > 0 \, | \, \alpha^{-1}x \in C\right\}$.
\end{definition}

\begin{exercice}
	$p(0) = 0$ quelque soit $C$.
\end{exercice}
%Donner des exercices et des exemples, avec les corrigés

Regardons quels résultats nous avons quand $C$ est lié à la topologie de $E$
normé.

\begin{proposition}
	Si $C$ est ouvert, alors $\exists M > 0$, $\forall x \in E$, $p(x) \leq M
	||x||$.
\end{proposition}

\ifdefined\outputproof
\begin{proof}

	Comme $C$ est ouvert et $0 \in C$ par hypothèse, il existe $B(0, r)
	\subseteq C$. Le cas $x = 0$ est trivial.

	Prenons $x \in E$ non nul, on a alors avec $\alpha^{-1} = \frac{r}{2} \frac{1}{||x||}$,
	$\alpha^{-1} x \in B(0, r)$, donc $\alpha^{-1} x \in C$, donc $p(x) \leq
	\alpha = \frac{2}{r} ||x||$.
\end{proof}
\fi

\begin{proposition}
	Si $C$ est ouvert et convexe, alors $C = \left\{x \in E \, | \, p(x) <
	1\right\}$.
\end{proposition}

\ifdefined\outputproof
\begin{proof}

\end{proof}
\fi

\begin{definition} [Hyperplan]
	Soit $H$ un sous-espace vectoriel de $E$. $H$ est appelé hyperplan
	(vectoriel) s'il est de codimension 1, c'est-à-dire que $E = H \oplus
	\mathbb{K}e$ où $e \in E$, non nul.
\end{definition}

\begin{proposition}
	$H$ est un hyperplan ssi $\exists \GSfunction{f}{E}{\mathbb{K}}$
	linéaire non nulle tel que $H = \ker(f)$. En d'autres termes, tout hyperplan
	est le noyau d'une forme linéaire non nulle.
\end{proposition}

\ifdefined\outputproof
\begin{proof}
	$(\Rightarrow)$ On a $E = H \oplus \mathbb{K}e$. Prenons
	$\GSfunction{f}{E}{\mathbb{K}}$ : $h + \lambda e \rightarrow \lambda$.
	On a bien $H = \ker(f)$.

	$(\Leftarrow)$ Soit $x \in E$, et $e \notin \ker(f)$. On a $x = (x - \lambda
	e) + \lambda e$.

	Choisissons un $\lambda$ tel que $x - \lambda e \in
	\ker(f)$. On a $f(x - \lambda e) = f(x) - \lambda f(e)$. Si $\lambda =
	\frac{f(x)}{f(e)}$, on a ce qu'on veut.
	On a donc bien que $\ker(f)$ est un hyperplan vectoriel car $E = \ker(f)
	\oplus \mathbb{K}e$.
\end{proof}
\fi

\begin{definition} [Hyperplan affine]
	Soit $\GSfunction{f}{E}{\mathbb{K}_{0}}$, et $\alpha \in \mathbb{K}$.

	Soit $H = \left\{ x \in E \, | \, f(x) = \alpha \right\}$.

	$H$ est appelé hyperplan (vectoriel) affine.
\end{definition}


\begin{definition}
	Soient $A$, $B \subseteq E$ convexes et disjoints.

	Soit $H = \left\{ x \in E \, | \, f(x) = \alpha\right\}$ un hyperplan affine

	On dit que \textbf{$H$ sépare $A$ et $B$ au sens large} si

	$\forall x \in A$, $f(x) \leq \alpha$ et $\forall x \in B$, $f(x) \geq
	\alpha$

	et on dit que \textbf{$H$ sépare $A$ et $B$ au sens strict} si

	$\exists \epsilon > 0$, $\forall x \in A$, $f(x) \leq \alpha -
	\epsilon$ et $\forall x \in B$, $f(x) \geq \alpha + \epsilon$.
\end{definition}

Nous en venons aux théorèmes de Hahn-Banach sous leurs formes géométriques.

\begin{theorem} [Théorème de Hahn-Banach 1 --- géométrique]
	Soit $A$, $B \subseteq E$ convexes et disjoints tel que $A$ est ouvert.

	Alors il existe un hyperplan vectoriel fermé qui sépare $A$ et $B$ au sens
	large.
\end{theorem}

\ifdefined\outputproof
\begin{proof}

\end{proof}
\fi

\begin{theorem} [Théorème de Hahn-Banach 2 --- géométrique]
	Soit $A$, $B \subseteq E$ convexes et disjoints tel que $A$ est fermé et $B$
	compact.

	Alors il existe un hyperplan vectoriel fermé qui sépare $A$ et $B$ au sens
	strict.
\end{theorem}

\ifdefined\outputproof
\begin{proof}

\end{proof}
\fi

\section{Applications géométriques}

\begin{corollary}
	Soient deux convexes $A$ et $B$ tel que $A-B$ est dense dans E.

	Alors $A$ et $B$ ne peuvent être séparés par un hyperplan fermé.
\end{corollary}

\ifdefined\outputproof
\begin{proof}
	Supposons qu'il existe un hyperplan fermé qui les sépare, c'est-à-dire qu'il
	existe $x^{*} \in E^{*}$ non nulle et $\alpha \in \mathbb{K}$ tel que
	$\forall a \in A, \, x^{*}(a) \leq \alpha$ et $\forall b \in B, \, x^{*}(b)
	\geq \alpha$.

	Soit $x \in A-B$ tel que $x^{*}(x) \neq 0$. Sans perte de généralité,
	$x^{*}(x) > 0$
	\footnote{Sinon, faire le même raisonnement avec $-x$. On aura
	bien $x^{*}(-x) = -x^{*}(x) > 0$.}.  On a qu'il existe une suite $(a_{n}
	- b_{n}) \rightarrow x$.

	On a $\forall n \in \naturel$, $x^{*}(a_{n}) \leq \alpha$ et $x^{*}(b_{n})
	\geq \alpha$, donc $x^{*}(a_{n} - b_{n}) = x^{*}(a_{n}) - x^{*}(b_{n}) \leq
	0$.

	Or, comme $x^{*}$ est continue, $x^{*}(a_{n} - b_{n}) \rightarrow x^{*}(x) >
	0$. Contradiction.
\end{proof}
\fi

\begin{corollary}
	Tout hyperplan vectoriel est soit fermé, soit dense dans $E$.
\end{corollary}

\ifdefined\outputproof
\begin{proof}

\end{proof}
\fi
